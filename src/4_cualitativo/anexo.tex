\documentclass[../ecuaciones_diferenciales.tex]{subfiles}

\begin{document}

\begin{figure}[ht]
  \centering
  \begin{subfigure}{0.25\textwidth}
    \centering
    \includegraphics{figuras/pdf/diagonal_dos_autovalores_positivos_a_gt_1.pdf}
    \caption*{Nodo inestable \\ \(|\lambda_1| < |\lambda_2|\) \\ \(\lambda_1, \lambda_2 > 0\)}
  \end{subfigure}%
  \begin{subfigure}{0.25\textwidth}
    \centering
    \includegraphics{figuras/pdf/diagonal_dos_autovalores_negativos_a_gt_1.pdf}
    \caption*{Nodo estable \\ \(|\lambda_1| < |\lambda_2|\) \\ \(\lambda_1, \lambda_2 < 0\)}
  \end{subfigure}%
  \begin{subfigure}{0.25\textwidth}
    \centering
    \includegraphics{figuras/pdf/diagonal_dos_autovalores_positivos_a_lt_1.pdf}
    \caption*{Nodo inestable \\ \(|\lambda_2| < |\lambda_1|\) \\ \(\lambda_1, \lambda_2 > 0\)}
  \end{subfigure}%
  \begin{subfigure}{0.25\textwidth}
    \centering
    \includegraphics{figuras/pdf/diagonal_dos_autovalores_negativos_a_lt_1.pdf}
    \caption*{Nodo estable \\ \(|\lambda_2| < |\lambda_1|\) \\ \(\lambda_1, \lambda_2 < 0\)}
  \end{subfigure}
  \caption{Nodos, se dan cuando \(\lambda_1\) y \(\lambda_2\) tienen el mismo signo.}
\end{figure}

\begin{figure}[ht]
  \centering
  \begin{subfigure}{0.25\textwidth}
    \centering
    \includegraphics{figuras/pdf/diagonal_dos_autovalores_a_eq_0_positivo_v1.pdf}
    \caption*{\(\lambda_1 > 0, \lambda_2 = 0\)}
  \end{subfigure}%
  \begin{subfigure}{0.25\textwidth}
    \centering
    \includegraphics{figuras/pdf/diagonal_dos_autovalores_a_eq_0_negativo_v1.pdf}
    \caption*{\(\lambda_1 < 0, \lambda_2 = 0\)}
  \end{subfigure}%
  \begin{subfigure}{0.25\textwidth}
    \centering
    \includegraphics{figuras/pdf/diagonal_dos_autovalores_a_eq_0_positivo_v2.pdf}
    \caption*{\(\lambda_1 = 0, \lambda_2 > 0\)}
  \end{subfigure}%
  \begin{subfigure}{0.25\textwidth}
    \centering
    \includegraphics{figuras/pdf/diagonal_dos_autovalores_a_eq_0_negativo_v2.pdf}
    \caption*{\(\lambda_1 = 0, \lambda_2 < 0\)}
  \end{subfigure}
  \caption{Rectas, se dan cuando algún autovalor es nulo.}
\end{figure}

\begin{figure}[ht]
  \centering
  \begin{subfigure}{0.25\textwidth}
    \centering
    \includegraphics{figuras/pdf/diagonal_dos_autovalores_a_lt_0_gt_lt.pdf}
    \caption*{Punto de silla \\ \(\lambda_1 > 0, \lambda_2 < 0\)}
  \end{subfigure}%
  \begin{subfigure}{0.25\textwidth}
    \centering
    \includegraphics{figuras/pdf/diagonal_dos_autovalores_a_lt_0_lt_gt.pdf}
    \caption*{Punto de silla \\ \(\lambda_1 < 0, \lambda_2 > 0\)}
  \end{subfigure}
  \caption{Puntos de silla, se dan cuando \(\lambda_1\) y \(\lambda_2\) tienen signo opuesto.}
\end{figure}

\begin{figure}[ht]
  \centering
  \begin{subfigure}{0.33\textwidth}
    \centering
    \includegraphics{figuras/pdf/diagonal_un_autovalor_l_gt_0.pdf}
    \caption*{Punto de estrella inestable \\ \(\lambda > 0\)}
  \end{subfigure}%
  \begin{subfigure}{0.33\textwidth}
    \centering
    \includegraphics{figuras/pdf/diagonal_un_autovalor_l_lt_0.pdf}
    \caption*{Punto de estrella estable \\ \(\lambda < 0\)}
  \end{subfigure}%
  \begin{subfigure}{0.33\textwidth}
    \centering
    \includegraphics{figuras/pdf/diagonal_un_autovalor_l_eq_0.pdf}
    \caption*{\(\lambda = 0\)}
  \end{subfigure}
  \caption{Puntos de estrella, se dan cuando \(J\) es diagonalizable con un único autovalor real.}
\end{figure}

\begin{figure}[ht]
  \centering
  \begin{subfigure}{0.33\textwidth}
    \centering
    \includegraphics{figuras/pdf/no_diagonal_l_gt_0.pdf}
    \caption*{Nodo impropio inestable \\ \(\lambda > 0\)}
  \end{subfigure}%
  \begin{subfigure}{0.33\textwidth}
    \centering
    \includegraphics{figuras/pdf/no_diagonal_l_lt_0.pdf}
    \caption*{Nodo impropio estable \\ \(\lambda < 0\)}
  \end{subfigure}%
  \begin{subfigure}{0.33\textwidth}
    \centering
    \includegraphics{figuras/pdf/no_diagonal_l_eq_0.pdf}
    \caption*{\(\lambda = 0\)}
  \end{subfigure}
  \caption{Nodos impropios, se dan cuando \(J\) es no diagonalizable con un único autovalor real.}
\end{figure}

\begin{figure}[ht]
  \begin{subfigure}{0.33\textwidth}
    \centering
    \includegraphics{figuras/pdf/autovalores_complejos_a_gt_0.pdf}
    \caption*{Foco inestable \\ \(a > 0\)}
  \end{subfigure}%
  \begin{subfigure}{0.33\textwidth}
    \centering
    \includegraphics{figuras/pdf/autovalores_complejos_a_lt_0.pdf}
    \caption*{Foco estable \\ \(a < 0\)}
  \end{subfigure}%
  \begin{subfigure}{0.33\textwidth}
    \centering
    \includegraphics{figuras/pdf/autovalores_complejos_a_eq_0.pdf}
    \caption*{Centro \\ \(a = 0\)}
  \end{subfigure}
  \caption{Focos, se dan cuando \(J\) es no diagonalizable con autovalores complejos conjugados \(\lambda = a \pm ib\).}
\end{figure}
\end{document}
