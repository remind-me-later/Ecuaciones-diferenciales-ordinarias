\documentclass[a4paper, 10pt, openany]{book}
%\documentclass[draft, a4paper, 10pt, openany]{book}
\usepackage[spanish]{babel}

\usepackage{parskip}
\usepackage{geometry}
\usepackage{multicol}

\usepackage{subfiles} 
\usepackage{tocbibind}
\usepackage[toc]{appendix}

\usepackage{tikz}
\usetikzlibrary{arrows, babel}
\usepackage[justification=centering]{caption}
\usepackage{subcaption}

\usepackage[shortlabels, inline]{enumitem}
\usepackage{csquotes}
\usepackage{hologo}

\usepackage{../sty/teoremas}
\usepackage{../sty/general}
\usepackage{../sty/conjuntos}
\usepackage{../sty/calculo}
\usepackage{../sty/algebra}

% Se carga el último 
\usepackage{hyperref}

\renewcommand\appendixpagename{Apéndices}
\renewcommand{\appendixtocname}{Apéndices}

\makeatletter
\renewcommand\mainmatter{
    \@mainmattertrue\cleardoublepage\renewcommand\thepage{\arabic{page}}}
\makeatother

\title{Elementos de ecuaciones diferenciales ordinarias}
\author{Ricardo Maurizio Paul \and Pablo Castellanos García}
\date{Curso 2020--2021}

\begin{document}

\frontmatter

\maketitle

\clearpage
%\cleardoublepage % openright
\vspace*{\fill}
\thispagestyle{empty} % Sin número de página
\begin{quotation}
\raggedleft
\em % énfasis
Duc, o parens celsique dominator poli, \\
Quocumque placuit; nulla parendi mora est. \\
Adsum inpiger. Fac nolle, comitabor gemens \\
Maiusque patiar, facere quod licuit bono. \\
Ducunt volentem fata, nolentem trahunt. \\
\bigskip
--- Seneca
\end{quotation}
\vspace*{\fill}

\tableofcontents

\chapter{Prefacio}
\subfile{prefacio}

\mainmatter

\chapter{Introducción}
\subfile{teoria/intro}

% Es necesario que las etiquetas de los capítulos estén en mayúsculas para que
% puedan ser referenciadas en el header de los apéndices :/

\chapter{Ecuaciones lineales de primer orden}
\label{CHAP:1}
\subfile{teoria/lineal}

\chapter{Sistemas de ecuaciones lineales de primer orden}
\label{CHAP:2}
\subfile{teoria/sistemas}

\chapter{Ecuación escalar de orden superior}
\label{CHAP:3}
\subfile{teoria/escalar}

\chapter{Análisis cualitativo de sistemas planos}
\label{CHAP:4}
\subfile{teoria/cualitativo}

\chapter{Series de potencias}
\label{CHAP:5}
\subfile{teoria/series}

\chapter{Transformada de Laplace}
\label{CHAP:6}
\subfile{teoria/laplace}

\begin{appendices}

\chapter{Problemas resueltos}

\section{Problemas del capítulo~\ref{CHAP:1}}
\subfile{problemas/lineal}

\section{Problemas del capítulo~\ref{CHAP:2}}
\subfile{problemas/sistemas}

\end{appendices}

\backmatter

\listoffigures

\listoftables

\begin{thebibliography}{1}
	\bibitem{introduction_real_analysis} 
		Bartle, Robert G., and Donald R. Sherbert.
		\textit{Introduction to Real Analysis}. 
		New York: Wiley, 2000. 
\end{thebibliography}

\end{document}
