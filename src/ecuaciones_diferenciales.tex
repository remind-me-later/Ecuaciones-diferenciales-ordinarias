\documentclass[a4paper, 10pt, openany]{book}


% \documentclass[draft, a4paper, 10pt, openany]{book}
\usepackage[a4paper]{geometry}

\usepackage[spanish, es-lcroman]{babel}

\usepackage{parskip}
\usepackage{multicol}

\usepackage{subfiles} 
\usepackage{tocbibind}
\usepackage[toc]{appendix}

\usepackage{graphicx}
\usepackage{tikz}
\usetikzlibrary{arrows, babel}
\usepackage[justification=centering]{caption}
\usepackage{subcaption}

\usepackage[shortlabels]{enumitem}
\usepackage{csquotes}

\usepackage{booktabs}

% \usepackage[Conny]{fncychap}

\usepackage{../sty/teoremas}
\usepackage{../sty/general}
\usepackage{../sty/conjuntos}
\usepackage{../sty/calculo}
\usepackage{../sty/algebra}

% \usepackage{xr}

% Se carga el último 
\usepackage{bookmark}

% Los apéndices se llaman apéndices
\renewcommand\appendixpagename{Apéndices}
\renewcommand{\appendixtocname}{Apéndices}

% mainmatter no reinicia el contador de páginas
\makeatletter
\renewcommand\mainmatter{
    \@mainmattertrue\cleardoublepage\renewcommand\thepage{\arabic{page}}}
\makeatother

\renewcommand{\thesubsection}{\thesection.\alph{subsection}}

\title{Ecuaciones diferenciales ordinarias}
\author{Ricardo Maurizio Paul \and Pablo Castellanos García}
\date{2020--2024}

\begin{document}

\frontmatter{}

\maketitle

\clearpage
%\cleardoublepage % openright
\vspace*{\fill}
\thispagestyle{empty} % Sin número de página
\begin{quotation}
	\raggedleft{}
	\em % énfasis
	Duc, o parens celsique dominator poli, \\
	Quocumque placuit; nulla parendi mora est. \\
	Adsum inpiger. Fac nolle, comitabor gemens \\
	Maiusque patiar, facere quod licuit bono. \\
	Ducunt volentem fata, nolentem trahunt. \\
	\bigskip
	--- Seneca
\end{quotation}
\vspace*{\fill}

\tableofcontents

\chapter{Prefacio}
\subfile{0_prefacio}

\mainmatter{}

% Es necesario que las etiquetas de los capítulos estén en mayúsculas para que
% puedan ser referenciadas en el header de los apéndices :/

\part{Ecuaciones lineales}

\chapter{Introducción}\label{CHAP:INTRO}
\subfile{1_intro/teoria}

\chapter{Ecuaciones lineales de primer orden}\label{CHAP:EC_LINEAL_PRIMER_ORDEN}
\subfile{2_ec_lin_primer_orden/teoria}
\newpage
\section{Ejercicios propuestos}
\subfile{2_ec_lin_primer_orden/problemas}

\chapter{Sistemas de ecuaciones lineales de primer orden}\label{CHAP:SISTEMAS}
\subfile{3_sistemas_lin/teoria}
\newpage
\section{Ejercicios propuestos}
\subfile{3_sistemas_lin/problemas}

\chapter{Análisis cualitativo de sistemas planos}\label{CHAP:CUALITATIVO}
\subfile{4_cualitativo/teoria}

\chapter{Series de potencias}\label{CHAP:SERIES}
\subfile{5_series/teoria}

\chapter{Transformada de Laplace}\label{CHAP:LAPLACE}
\subfile{6_laplace/teoria}

\part{Ecuaciones no lineales}

\chapter{Teoría fundamental}\label{CHAP:TEORIA_FUNDAMENTAL}
\subfile{7_teoria_fundamental/teoria}

% \part{Apéndices}
% \appendix

% \chapter{Problemas resueltos}

% \section{Problemas del capítulo~\ref{CHAP:EC_LINEAL_PRIMER_ORDEN}}
% \subfile{2_ec_lin_primer_orden/problemas}

% \section{Problemas del capítulo~\ref{CHAP:SISTEMAS}}
% \subfile{3_sistemas_lin/problemas}

% \chapter{Resumen de diagramas de fase}
% \subfile{4_cualitativo/anexo}

\backmatter{}

\listoffigures

% \listoftables

\bibliographystyle{apalike}
\bibliography{bibliografia}
\nocite{*}

\end{document}
