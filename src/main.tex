%\documentclass[a4paper, 10pt, openany]{book}
\documentclass[draft, a4paper, 10pt, openany]{book}
\usepackage[spanish]{babel}

\usepackage{parskip}
\usepackage{geometry}
\usepackage{multicol}

\usepackage{subfiles} 
\usepackage{tocbibind}
\usepackage[toc]{appendix}

\usepackage{graphicx}
\usepackage{tikz}
\usetikzlibrary{arrows, babel}
\usepackage[justification=centering]{caption}
\usepackage{subcaption}

\usepackage[shortlabels]{enumitem}
\usepackage{csquotes}
\usepackage{hologo}

\usepackage{../sty/teoremas}
\usepackage{../sty/general}
\usepackage{../sty/conjuntos}
\usepackage{../sty/calculo}
\usepackage{../sty/algebra}

% Se carga el último 
\usepackage{hyperref}

% Los apéndices se llaman apéndices
\renewcommand\appendixpagename{Apéndices}
\renewcommand{\appendixtocname}{Apéndices}

% mainmatter no reinicia el contador de páginas
\makeatletter
\renewcommand\mainmatter{
    \@mainmattertrue\cleardoublepage\renewcommand\thepage{\arabic{page}}}
\makeatother

\renewcommand{\thesubsection}{\thesection.\alph{subsection}}

\title{Elementos de ecuaciones diferenciales ordinarias}
\author{Ricardo Maurizio Paul \and Pablo Castellanos García}
\date{Curso 2020--2021}

\begin{document}

\frontmatter

\maketitle

\clearpage
%\cleardoublepage % openright
\vspace*{\fill}
\thispagestyle{empty} % Sin número de página
\begin{quotation}
\raggedleft
\em % énfasis
Duc, o parens celsique dominator poli, \\
Quocumque placuit; nulla parendi mora est. \\
Adsum inpiger. Fac nolle, comitabor gemens \\
Maiusque patiar, facere quod licuit bono. \\
Ducunt volentem fata, nolentem trahunt. \\
\bigskip
--- Seneca
\end{quotation}
\vspace*{\fill}

\tableofcontents

\chapter{Prefacio}
\subfile{prefacio}

\mainmatter

% Es necesario que las etiquetas de los capítulos estén en mayúsculas para que
% puedan ser referenciadas en el header de los apéndices :/

\chapter{Introducción}
\label{CHAP:INTRO}
\subfile{intro/teoria}

\chapter{Ecuaciones lineales de primer orden}
\label{CHAP:ORDEN1}
\subfile{orden1/teoria}

\chapter{Sistemas de ecuaciones lineales de primer orden}
\label{CHAP:SISTEMAS}
\subfile{sistemas/teoria}

\chapter{Análisis cualitativo de sistemas planos}
\label{CHAP:CUALITATIVO}
\subfile{cualitativo/teoria}

\chapter{Series de potencias}
\label{CHAP:SERIES}
\subfile{series/teoria}

\chapter{Transformada de Laplace}
\label{CHAP:LAPLACE}
\subfile{laplace/teoria}

\begin{appendices}

\chapter{Problemas resueltos}

\section{Problemas del capítulo~\ref{CHAP:ORDEN1}}
\subfile{orden1/problemas}

\section{Problemas del capítulo~\ref{CHAP:SISTEMAS}}
\subfile{sistemas/problemas}

\end{appendices}

\backmatter

\listoffigures

\listoftables

\begin{thebibliography}{9}
	\bibitem{introduction_real_analysis} 
		Bartle, Robert G. \& Sherbert, Donald R.
		\textit{Introduction to Real Analysis}. 
		New York: Wiley, 2000. 
    \bibitem{ecuaciones_lineales}
        Fernández Pérez, Carlos.
        \textit{Ecuaciones diferenciales I: ecuaciones lineales}.
        Pirámide, 1992.
\end{thebibliography}

\end{document}
