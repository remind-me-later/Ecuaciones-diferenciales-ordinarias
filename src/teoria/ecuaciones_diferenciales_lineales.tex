\documentclass[../main.tex]{subfiles}

\begin{document}

Comenzamos estudiando las ecuaciones diferenciales mas sencillas, las ecuaciones
lineales de primer orden, como veremos, son siempre resolubles, esto es existe
una fórmula dependiente de un solo parámetro que proporciona todo el conjunto de
soluciones.

\begin{definition}
Las ecuaciones diferenciales lineales de primer orden son aquellas de la forma
\[x' = a(t)x + b(t)\]
donde \(a\) y \(b\) son continuas en \((\alpha, \omega) \subset \R\).
\end{definition}

Antes de resolver estas ecuaciones, centraremos nuestra atención en un problema
más simple, la obtención de soluciones de una ecuación lineal homogénea de
primer orden asociada a la ecuación original.

\section{Ecuación diferencial lineal homogénea de primer orden} \label{sec:hom}

\begin{definition}
Una ecuación diferencial lineal homogénea de primer orden es una ecuación de la
forma
\[x' = a(t)x.\]
con \(a\) continua en \((\alpha, \omega) \subset \R\).
\end{definition}

Buscamos ahora las soluciones de la ecuación homogénea. Supongamos que la
solución no se anula en ningún punto, entonces por el teorema de Bolzano
\(x > 0\) o \(x < 0\) en \((\alpha, \omega)\). En este caso recurrimos a un
método que llamaremos separación de variables, por el teorema fundamental del
cálculo podemos escribir
\[\frac{x'}{x} = a(t) \iff \log\abs{x(t)} = \int_{\beta}^t a(s) \dif s + c,
	\quad \beta \in (\alpha, \omega).\]
Para aligerar la notación escribiremos \(\int^t a(s) \dif s\), para referirnos
a cualquier primitiva de \(a\). Por lo anterior
\[x(t) = k e^{\int^t a(s) \dif s}.\]

Vamos a demostrar ahora que la solución obtenida es la solución general de la
ecuación homogénea.

\begin{theorem}
	Todas las soluciones de una ecuación diferencial lineal homogénea,
	\(x' = a(t)x\), son de la forma
	\begin{equation}
		x(t) = k e^{\int^t a(s) \dif s}. \label{eq:sol_gen}
	\end{equation}
\end{theorem}

\begin{proof}
	Primero probamos que \(k e^{\int^t a(s) \dif s}\) es
	solución de la ecuación homogénea, para ello basta derivar
	en~\eqref{eq:sol_gen}
	\[x'(t) = k a(t) e^{\int^t a(s) \dif s} = a(t) x(t).\]

	Probamos ahora que no existen más soluciones, para ello, sea \(u(t)\) una
	solución arbitraria de la ecuación, consideramos la función
	\[\Phi(t) = u(t) e^{-\int^t a(s) \dif s}\]
	y calculamos su derivada obteniendo
	\begin{align*}
		\Phi'(t) &= u'(t) e^{-\int^t a(s) \dif s}
			- u(t) a(t) e^{-\int^t a(s) \dif s} \\
			&= a(t) u(t) e^{-\int^t a(s) \dif s}
			- u(t) a(t) e^{-\int^t a(s) \dif s}
			= 0.
	\end{align*}

	Luego la función \(\Phi\) es constante, por lo que
	\(u(t) = k e^{\int^t a(s) \dif s}\).
\end{proof}

\begin{example}
	Obtener la solución general de la ecuación \(y' + 2ty = 0\).
\end{example}

\begin{solution}
	Utilizando la técnica de separación de variables vista anteriormente
	\[\frac{y'}{y} = -2t \iff \log\abs{y} = -t^2 + c \iff y = k e^{-t^2}\]
\end{solution}

\subsection{Problema del valor inicial} \label{sec:pvi}

Vemos cómo se plantea el problema del valor inicial en el caso de ecuaciones
homogéneas. Tenemos una ecuación lineal homogénea de primer orden, esto es
\(x' = a(t)x\) y además queremos que \(x(t_0) = x_0\) para algún
\(t_0 \in (\alpha, \omega)\) y \(x_0 \in \R\).
Sabemos que la solución general de la ecuación es de la forma~\eqref{eq:sol_gen}
y podemos usar la libertad que tenemos para elegir la primitiva para ponerla como
\[x(t) = k e^{\int_{t_0}^t a(s) \dif s},\]
ahora queremos que \(x(t_0) = x_0\), para ello notamos que
\[x_0 = x(t_0) = k e^{\int_{t_0}^{t_0} a(s) \dif s} = k,\]
hemos probado así que el problema del valor inicial tiene solución única,
dada por
\begin{equation}
	x(t) = x_0 e^{\int_{t_0}^t a(s) \dif s} \label{eq:sol_pvi}
\end{equation}
y además la solución está bien definida en el intervalo \((\alpha, \omega)\).

\subsection{Estructura del espacio de soluciones}

Observamos que el conjunto de soluciones de una ecuación lineal homogénea de
primer orden es un espacio vectorial de dimensión 1, o más precisamente un
subespacio vectorial de dimensión 1 de \(C^1(\alpha, \omega)\), es decir, el
espacio vectorial de funciones de clase \(C^1\) en el intervalo
\((\alpha, \omega) \subset \R\).

Vamos a ver cómo esto se deduce de la linealidad de la ecuación. Para ello
definimos el operador diferencial:
\begin{align*}
  L : C^1(\alpha, \omega) &\to C(\alpha, \omega) \\
  x &\mapsto \frac{d x}{\dif t} - a(t)x
\end{align*}

Recordamos que decimos que una función es lineal cuando cumple
\[f(ax + by) = af(x) + bf(y), \quad a, b \in \mathbb{K},\]
entonces la linealidad de \(L\) se deduce de la linealidad de todas las
operaciones dentro de su definición.

\begin{remark}
La linealidad de la ecuación diferencial equivale a la linealidad de
\(L\) y, además, \(x\) es solución de \(x' = a(t)x\) si y solo si
\(x \in \text{ker}(L)\). Sabemos del álgebra lineal que el núcleo de un
homomorfismo vectorial es siempre un subespacio vectorial del espacio de dominio
y, por tanto, el conjunto de soluciones de la ecuación homogénea es un espacio
vectorial.
\end{remark}

\section{El operador diferencial}

Dada una ecuación diferencial lineal homogénea de la forma
\[x^{(n)}(t) = a_{n-1}x^{(n - 1)}(t) + \dots + a_0(t)x(t).\]

Podemos asociar a esta ecuación un operador diferencial
\[L = \frac{d^n}{\dif t^n} - a_{n-1}(t)\frac{d^{n-1}}{\dif t^{n-1}}
	- \dots - a_1(t)\frac{d}{\dif t} - a_0(t)\]
donde el término \(k\)-ésimo \(a_k(t) d^k/\dif t^k\) actúa sobre una
función \(u(t)\) como
\[L(u(t)) = a_k(t) \frac{d^k}{\dif t^k} u(t) = a_k(t) u^{(k)}(t).\]

Como derivar es lineal:
\[\frac{d}{\dif t} (\alpha u(t) + \beta v(t))
	= \alpha \frac{d}{\dif t} u(t) + \beta \frac{d}{\dif t} v(t)\]
y multiplicar por una función fija también es lineal
\[a(t)(\alpha u(t) + \beta v(t)) = \alpha a(t) u(t) + \beta a(t) u(t),\]
el operador composición de las anteriores \(a_k(t) d^k/\dif t^k\) es
lineal. Tenemos así que \(L\) es lineal como operador de tipo
\(L : C^n(\alpha, \omega) \to C(\alpha, \omega)\), asumiendo que
\(a_k(t) \in C(\alpha, \omega)\) para todo \(k=0, \dots, n-1\).

\begin{remark}
	Evidentemente tenemos que las soluciones de una ecuación diferencial
	homogénea corresponden a las funciones tal que \(L(x(t)) = 0\). Por lo que
	el conjunto de soluciones es el núcleo del operador \(L\) así definido.
\end{remark}

\begin{remark}
	En general tenemos que \(x(t)\) es solución de una ecuación diferencial
	general si y solo si \(L(x(t)) = b(t)\), donde hemos definido \(L\)
	mediante la ecuación homogénea asociada. Por lo que el espacio de
	soluciones de una ecuación general lineal es un espacio afín de dimensión
	\(n\).
\end{remark}

\section{Ecuación diferencial lineal de primer orden}

Ahora que ya hemos obtenido la solución general~\autoref{sec:hom} para
ecuaciones lineales homogéneas y también su solución única en el caso del
problema del valor inicial~\autoref{sec:pvi}, podemos afrontar el problema de
obtener las soluciones de una ecuación diferencial lineal:
\[x' = a(t)x + b(t),\]

Consideramos el operador diferencial asociado a su ecuación homogénea
\(L = \frac{d}{\dif t} - a(t)\).

\begin{lemma}
	si \(x_0\) es solución de la ecuación homogénea asociada a una ecuación
	lineal y \(x_p\) es solución de la ecuación entonces
	\[x_0 + x_p\]
	es solución de la ecuación lineal.
\end{lemma}

\begin{proof}
	Por la linealidad del operador diferencial
	\[L(x_0(t) + x_p(t)) = L(x_0(t)) + L(x_p(t)) = b(t).\]
\end{proof}

\begin{lemma}
	si \(x_1\) y \(x_2\) son soluciones de una ecuación diferencial lineal
	entonces \(x_1 - x_2\) es también solución.
\end{lemma}

\begin{proof}
	Por la linealidad del operador diferencial
	\[L(x_1 - x_2) = L(x_1) - L(x_2) = b(t) - b(t) = 0.\]
\end{proof}

\begin{corollary}
	La solución general de una ecuación diferencial lineal de primer orden es
	\[x_p(t) + x_n(t)\]
	donde \(x_p(t)\) es una solución arbitraria de la ecuación y \(x_n(t)\) es
	la solución general de la ecuación homogénea asociada.
\end{corollary}

Nos proponemos ahora obtener una solución de forma explícita, haremos esto
mediante el método de variación de constantes.

\subsection{Método de variación de constantes}

Conocemos una forma explícita de calcular una solución de la ecuación homogénea
asociada a una ecuación lineal~\autoref{sec:hom}. Conjeturamos que hay una
solución de la ecuación lineal de la forma
\begin{equation}\label{xp}
  x_p(t) = k(t) e^{\int^t a(s) \dif s}
\end{equation}
e intentamos obtener \(k(t)\). Buscamos entonces que se cumpla la igualdad
\[x'_p(t) = a(t)x_p(t) + b(t) = a(t)k(t) e^{\int^t a(s) \dif s} + b(t),\]
podemos derivar en (\ref{xp}) obteniendo
\[x'_p(t) = k'(t) e^{\int^t a(s) \dif s} + k(t)a(t) e^{\int^t a(s) \dif s}\]
y simplemente igualando y despejando hallamos la derivada \(k'\):
\[k'(t) = b(t)e^{-\int^t a(s) \dif s}.\]

Tenemos así que la solución general de la ecuación diferencial es
\[x(t) = \parens{k + \int^t b(s) e^{-\int^s a(u) \dif u} \dif s}
	e^{\int^t a(s) \dif s}.\]

\begin{remark}
	No es necesario aprender esta ecuación de memoria, se utiliza el método de
	variación de constantes para obtener formas explícitas para cada problema en
	particular.
\end{remark}

\subsection{Problema del valor inicial}

Como de costumbre consideramos la ecuación diferencial lineal junto a una
condición \(x(t_0) = x_0\), con \(t_0 \in (\alpha, \omega)\) y \(x_0 \in \R\).
Consideramos la solución general fijando el límite inferior a \(t_0\),
\[x(t) = \parens{k + \int_{t_0}^t b(s) e^{-\int_{t_0}^s a(u) \dif u} \dif s}
	e^{\int_{t_0}^t a(s) \dif s}\]

Sin más que sustituir obtenemos que \(k = x_0\), por lo que nuestra ecuación
tiene solución única.

\begin{theorem}
	El problema del valor inicial
	\[
	\begin{cases}
		x = a(t)x + b(t), & a, b \in C(\alpha, \omega) \\
		x(t_0) = x_0, 	  & t_0 \in (\alpha, \omega), \quad x_0 \in \R
	\end{cases}
	\]
	tiene solución única dada por:
	\[x(t) =
		\parens{x_0 + \int_{t_0}^t b(s) e^{-\int_{t_0}^s a(u) \dif u} \dif s}
		e^{\int_{t_0}^t a(s) \dif s}\]
\end{theorem}

Las gráficas de las soluciones ``rellenan'' la franja
\((\alpha, \omega) \times \R\) sin cortarse. La rellenan porque para cualquier
punto \((t_0, x_0) \in (\alpha, \omega) \times \R\) hay una solución que pasa
por él, y no se cortan entre sí porque sólo hay una solución que pase por cada
punto, como consecuencia del teorema anterior.

\begin{example}
	Resuélvase el problema del valor inicial:
	\[
	\begin{cases}
		x' + \frac{x}{t} = \cos t \\
		x(\pi) = 1
	\end{cases}
	\]
\end{example}

\begin{solution}
	Observamos que el intervalo donde vamos a resolver el problema es
	\((0, +\infty)\). Puesto que \(\pi\) ha de pertenecer al intervalo
	\(t \neq 0\). La ecuación homogénea asociada es \(x' + x/t = 0\),
	separando variables y calculando primitivas a ambos lados obtenemos
	\[\frac{x'}{x} = -\frac{1}{t} \iff \log\abs{x}
		= -\log\abs{t} + c \iff x = \frac{k}{t}.\]
	Utilizamos ahora la técnica de variación de constantes, consideramos
	\(x_p(t) = k(t)/t\) e imponemos que es solución de la ecuación diferencial,
	por tanto
	\[x'_p(t) = \frac{k'(t)}{t} - \frac{k(t)}{t^2}\]
	asimismo
	\[\cos t - \frac{x_p(t)}{t} = \cos t - \frac{k(t)}{t^2}\]
	por lo que \(k'(t)/t = \cos t\). Basta elegir entonces \(k(t)\) igual a una
	primitiva de \(t \cos t\), integrando por partes
	\(k(t) = t \sin t + \cos t\). Recordamos que la solución general de la
	ecuación diferencial es la suma de la solución general de la homogénea más
	la solución particular obtenida
	\[x(t) = \frac{1}{t}(k + t \sin t + \cos t).\]

	Calculamos \(k\) usando la condición inicial \(x(\pi) = 1\),
	\[(k - 1)\frac{1}{\pi} = 1 \iff k = \pi + 1.\]

	Sustituimos en la solución general obteniendo así la solución única a
	nuestro problema
	\[x(t) = \frac{1}{t}(\pi + 1 + t \sin t + \cos t).\]
\end{solution}

\section{Propagación de enfermedades}

El pionero en la modelización de la propagación de enfermedades mediante
ecuaciones diferenciales fue Ronald Ross (Premio Nobel de Medicina 1915). Los 
grandes avances en esta disciplina comienzan cuando Kermack y McKendick
proponen el modelo SIR, lo que se conoce como el inicio de la epidemiología
moderna.

\subsection{Modelos simples para propagación de enfermedades}

Una enfermedad contagiosa se propaga a través de la interacción entre personas
enfermas y sanas. Denotaremos \(x(t)\) al número de persona enfermas a tiempo 
\(t\), asumimos \(x(t) \in \R\), y denotaremos \(y(t)\) al número de personas
sanas. Asumiremos que solo existen personas sanas o enfermas, el SIR también
incluye personas inmunizadas y muertos, que hacen variar el total de personas en
nuestro modelo. Haremos un par de hipótesis para simplificar el modelo:
\begin{enumerate}[i)]
	\item \(x'\) depende del número de interacciones entre personas sanas y 
		enfermas.

	\item El susodicho número de interacciones es proporcional a \(xy\).
\end{enumerate}

La última hipótesis es debida a Hamer, quién la usó con éxito para modelar
la propagación del Saranpión en Inglaterra en 1905. Tenemos así un primer
modelo sencillo, en una población de \(n\) individuos:
\[x' = k x(n - x).\]

Veremos que dependiendo de \(k\) la enfermedad desaparecerá o se hará endémica. 
Otra observación es que la ecuación es no lineal, por lo que se escapa de
nuestra ``zona de comfort", pero podremos resolverla fácilmente separando
variables.

\begin{remark}
	La misma ecuación, conocida como ecuación logística, aparece también en
	modelos poblacionales.
\end{remark}

Tal como hemos planteado la ecuación modela solo enfermedades sin cura, por
ejemplo el SIDA, ya que no contemplamos que un paciente pueda pasar de enfermo a
sano. Se puede incluir en el modelo una tasa de recuperación, quedaría entonces:
\[x' = k x(n - x) - \beta x,\]
donde \(\beta\) es la probabilidad de recuperación, independiente de factores
poblacionales. Este es el modelo más simple con nombre, el modelo SIS, el cual
modela enfermedades que no generan inmunidad. Este modelo posee una cualidad
importante del modelo SIS, el teorema del umbral, que enuncia que en función de 
los parámetros la enfermedad desaparece o se hace endémica.

\subsection{El modelo más simple}

Nos proponemos resolver la ecuación:
\[x' = k x(n - x),\]
lo haremos mediante la técnica de las variables separadas. Para ello notamos que
el número de personas infectadas y sanas es siempre mayor o igual que 0, en caso
de que alguno sea cero no hay evolución, por lo que asumimos que \(0 < x < n\).
Separando las variables nos queda
\[\frac{x'}{x(n - x)} = k.\]
Para poder continuar debemos obtener una primitiva del término a la izquierda,
separamos en fracciones parciales
\[\frac{1}{y(n - y)} = \frac{A}{y} + \frac{B}{n - y} 
	= \frac{y(B - A) + nA}{y(n - y)} \implies A = B = \frac{1}{n}.\]

La primitiva que buscamos es pues:
\[\frac{1}{n} (\log(y) - \log(n - y)) = \frac{1}{n} \log \frac{y}{n - y}.\]

Sustituyendo y tomando primitivas a ambos lados de la ecuación original
\[\frac{1}{n} \log\frac{x}{n - x} = kt + c 
	\implies x = \frac{n e^{nc} e^{nkt}}{1 + e^{nc} e^{nkt}}
	\implies x = \frac{n}{1 + c' e^{-nkt}},\]
donde \(c'\) es una constante positiva. El conjunto de soluciones no es un
espacio vectorial. Según este modelo la pandemia infecta a toda la población,
\(\lim_{t \to \infty} x(t) = n\). En la práctica, la evolución de una pandemia 
depende de las medidas que se tomen para evitar la infección, por lo que modelos
más refinados modelan la probabilidad de contagio en función del tiempo.

\end{document}
