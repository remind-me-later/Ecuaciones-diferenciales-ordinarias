\documentclass[../main.tex]{subfiles}

\begin{document}

Comenzamos estudiando las ecuaciones diferenciales mas sencillas, las ecuaciones
lineales de primer orden, como veremos, son siempre resolubles, esto es existe
una fórmula dependiente de un solo parámetro que proporciona todo el conjunto de
soluciones.

\begin{definition}
Las ecuaciones diferenciales lineales de primer orden son aquellas de la forma
\[x' = a(t)x + b(t)\]
donde \(a\) y \(b\) son continuas en \((\alpha, \omega) \subset \R\).
\end{definition}

Antes de resolver estas ecuaciones, centraremos nuestra atención en un problema
más simple, la obtención de soluciones de una ecuación lineal homogénea de 
primer orden asociada a la ecuación original.

\section{Ecuación diferencial lineal homogénea de primer orden}

\begin{definition}
Una ecuación diferencial lineal homogénea de primer orden es una ecuación de la
forma
\[x' = a(t)x.\]
con \(a\) continua en \((\alpha, \omega) \subset \R\).
\end{definition}

Buscamos ahora las soluciones de la ecuación homogénea. Supongamos que la
solución no se anula en ningún punto, entonces por el teorema de Bolzano
\(x > 0\) o \(x < 0\) en \((\alpha, \omega)\). En este caso recurrimos a un
método que llamaremos separación de variables, por el teorema fundamental del
cálculo podemos escribir
\[\frac{x'}{x} = a(t) \iff \log\abs{x(t)} = \int_{\beta}^t a(s) \dif s + c,
	\quad \beta \in (\alpha, \omega).\]
Para aligerar la notación escribiremos \(\int^t a(s) \dif s\), para referirnos 
a cualquier primitiva de \(a\). Por lo anterior
\[x(t) = k e^{\int^t a(s) \dif s}.\]

Vamos a demostrar ahora que la solución obtenida es la solución general de la 
ecuación homogénea.

\begin{theorem}
	Todas las soluciones de una ecuación diferencial lineal homogénea,
	\(x' = a(t)x\), son de la forma
	\begin{equation}
		x(t) = k e^{\int^t a(s) \dif s}. \label{eq:sol_gen}
	\end{equation}
\end{theorem}

\begin{proof}
	Primero probamos que \(k e^{\int^t a(s) \dif s}\) es
	solución de la ecuación homogénea, para ello basta derivar 
	en~\eqref{eq:sol_gen}
	\[x'(t) = k a(t) e^{\int^t a(s) \dif s} = a(t) x(t).\]

	Probamos ahora que no existen más soluciones, para ello, sea \(u(t)\) una
	solución arbitraria de la ecuación, consideramos la función
	\[\Phi(t) = u(t) e^{-\int^t a(s) \dif s}\]
	y calculamos su derivada obteniendo
	\begin{align*}
		\Phi'(t) &= u'(t) e^{-\int^t a(s) \dif s}
			- u(t) a(t) e^{-\int^t a(s) \dif s} \\
			&= a(t) u(t) e^{-\int^t a(s) \dif s} 
			- u(t) a(t) e^{-\int^t a(s) \dif s}
			= 0.
	\end{align*}

	Luego la función \(\Phi\) es constante, por lo que 
	\(u(t) = k e^{\int^t a(s) \dif s}\).
\end{proof}

\begin{example}
	Obtener la solución general de la ecuación \(y' + 2ty = 0\).
\end{example}

\begin{solution}
	Utilizando la técnica de separación de variables vista anteriormente
	\[\frac{y'}{y} = -2t \iff \log\abs{y} = -t^2 + c \iff y = k e^{-t^2}\]
\end{solution}

\subsection{Problema del valor inicial}

Vemos cómo se plantea el problema del valor inicial en el caso de ecuaciones
homogéneas. Tenemos una ecuación lineal homogénea de primer orden, esto es
\(x' = a(t)x\) y además queremos que \(x(t_0) = x_0\) para algún 
\(t_0 \in (\alpha, \omega)\) y \(x_0 \in \R\). 
Sabemos que la solución general de la ecuación es de la forma~\eqref{eq:sol_gen}
y podemos usar la libertad que tenemos para elegir la primitiva para ponerla como
\[x(t) = k e^{\int_{t_0}^t a(s) \dif s},\]
ahora queremos que \(x(t_0) = x_0\), para ello notamos que
\[x_0 = x(t_0) = k e^{\int_{t_0}^{t_0} a(s) \dif s} = k,\]
hemos probado así que el problema del valor inicial tiene solución única, 
dada por
\begin{equation}
	x(t) = x_0 e^{\int_{t_0}^t a(s) \dif s} \label{eq:sol_pvi}
\end{equation}
y además la solución está bien definida en el intervalo \((\alpha, \omega)\).

\subsection{Estructura del espacio de soluciones}

Observamos que el conjunto de soluciones de una ecuación lineal homogénea de
primer orden es un espacio vectorial de dimensión 1, o más precisamente un
subespacio vectorial de dimensión 1 de \(C^1(\alpha, \omega)\), es decir, el
espacio vectorial de funciones de clase \(C^1\) en el intervalo
\((\alpha, \omega) \subset \R\).

Vamos a ver cómo esto se deduce de la linealidad de la ecuación. Para ello
definimos el operador diferencial:
\begin{align*}
  L : C^1(\alpha, \omega) &\to C(\alpha, \omega) \\
  x &\mapsto \frac{\dif x}{\dif t} - a(t)x
\end{align*}

Recordamos que decimos que una función es lineal cuando cumple
\[f(ax + by) = af(x) + bf(y), \quad a, b \in \mathbb{K},\]
entonces la linealidad de \(L\) se deduce de la linealidad de todas las
operaciones dentro de su definición.

\begin{remark}
La linealidad de la ecuación diferencial equivale a la linealidad de
\(L\) y, además, \(x\) es solución de \(x' = a(t)x\) si y solo si
\(x \in \text{ker}(L)\). Sabemos del álgebra lineal que el núcleo de un
homomorfismo vectorial es siempre un subespacio vectorial del espacio de dominio
y, por tanto, el conjunto de soluciones de la ecuación homogénea es un espacio
vectorial.
\end{remark}

\section{El operador diferencial}

Partimos de una ecuación diferencial lineal homogénea:
\[x^{(n)}(t) = a_{n-1}x^{(n - 1)}(t) + \dots + a_0(t)x(t).\]

Podemos asociar a esta ecuación un operador diferencial
\[L = \frac{\dif^n}{\dif t^n} - a_{n-1}(t)\frac{\dif^{n-1}}{\dif t^{n-1}}
	- \dots - a_1(t)\frac{\dif}{\dif t} - a_0(t)\]
donde el último término \(a_k(t) \dif^k/\dif t^k\) actúa sobre una 
función \(u(t)\) como
\[L(u(t)) = a_k(t) \frac{\dif^k}{\dif t^k} u(t) = a_k(t) u^{(k)}(t).\]

Como derivar es lineal:
\[\frac{\dif}{\dif t} (\alpha u(t) + \beta v(t)) 
	= \alpha \frac{\dif}{\dif t} u(t) + \beta \frac{\dif}{\dif t} v(t)\]
y multiplicar por una función fija también es lineal
\[a(t)(\alpha u(t) + \beta v(t)) = \alpha a(t) u(t) + \beta a(t) u(t),\]
el operador composición de las anteriores \(a_k(t) \dif^k/\dif t^k\) es
lineal. Tenemos así que \(L\) es lineal como operador de tipo
\(L : C^n(\alpha, \omega) \to C(\alpha, \omega)\), asumiendo que
\(a_k(t) \in C(\alpha, \omega)\) para todo \(k\) natural.

Evidentemente tenemos que las soluciones de ecuación diferencial 
corresponden a las funciones tal que \(L(x(t)) = 0\). Por lo que
el conjunto de soluciones es el núcleo del operador \(L\) así definido.

En general tenemos que \(x(t)\) es solución de una ecuación diferencial
general si y solo si \(L(x(t)) = b(t)\), donde hemos definido \(L\)
mediante la ecuación homogénea asociada. Por lo que el espacio de 
soluciones de una ecuación general lineal es un espacio afín de dimensión \(n\).

\section{Ecuación diferencial lineal de primer orden}

Ahora que ya hemos obtenido la solución general~\eqref{eq:sol_gen} para 
ecuaciones lineales homogéneas y también su solución única en el caso del 
problema del valor inicial~\eqref{eq:sol_pvi}, podemos afrontar el problema de
obtener las soluciones de una ecuación diferencial lineal:
\[x' = a(t)x + b(t),\]

Consideramos el operador diferencial asociado a su ecuación homogénea 
\(L = \frac{\dif}{\dif t} - a(t)\). Observamos que si \(x_0(t)\) es solución 
de la ecuación homogénea y \(x_p(t)\) es solución de la ecuación lineal entonces
\(x_0(t) + x_p(t)\) es solución de la ecuación lineal. Esto se debe a que
\[L(x_0(t) + x_p(t)) = L(x_0(t)) + L(x_p(t)) = b(t).\]

Además, si \(x_1\) y \(x_2\) son soluciones de la ecuación entonces 
\(x_1 - x_2\) es también solución. Lo demostramos:
\[L(x_1 - x_2) = L(x_1) - L(x_2) = b(t) - b(t) = 0.\]

\begin{theorem}
	La solución general de una ecuación diferencial lineal de primer orden es 
	\[x_p(t) + x_n(t)\]
	donde \(x_p(t)\) es una solución arbitraria de la ecuación y \(x_n(t)\) es
	la solución general de la ecuación homogénea asociada.
\end{theorem}

Nos proponemos ahora obtener una solución de forma explícita, haremos esto
mediante el método de variación de constantes.

\subsection{Método de variación de constantes}

Conocemos una forma explícita de calcular una solución de la ecuación homogénea
asociada a una ecuación lineal
\[x(t) = k e^{\int^t a(s) \dif s}.\]

Conjeturamos que hay una solución de la ecuación lineal de la forma
\[x_p(t) = k(t) e^{\int^t a(s) \dif s}\]
e intentamos obtener \(k(t)\). Buscamos entonces que se cumpla la igualdad
\[x'_p(t) = a(t)x_p(t) + b(t) = a(t)k(t) e^{\int^t a(s) \dif s} + b(t),\]
podemos entonces derivar en la primera fórmula obteniendo
\[k'(t) e^{\int^t a(s) \dif s} + k(t)a(t) e^{\int^t a(s) \dif s}\]
simplemente sustituyendo y reordenando obtenemos ahora 
\[k'(t) = b(t)e^{-\int^t a(s) \dif s}.\]

Tenemos así que la solución general de la ecuación diferencial es 
\[x(t) = \parens{k + \int^t b(s) e^{-\int^s a(u) \dif u} \dif s} 
	e^{\int^t a(s) \dif s}.\]

\end{document}
