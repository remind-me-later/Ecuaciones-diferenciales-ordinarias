\documentclass[../main.tex]{subfiles}

\begin{document}

Sean \(x_1, x_2, \dots, x_n \in C^1(\alpha, \omega)\) variables, las cuales son
funciones de clase \(C^1\) sobre un intervalo de \(\R\) y \(a_{ij}\), \(b_i\)
funciones continuas en \((\alpha, \omega)\) para todo \(i, j \in \N\).

\begin{definition}
	Un sistema de ecuaciones diferenciales de primer orden es un sistema de la
	forma:
	\[\eqsys{
			x'_1 &= f_1(t, x_1, x_2, \dots, x_n) \\
			x'_2 &= f_2(t, x_1, x_2, \dots, x_n) \\
				 &\vdots \\
			x'_n &= f_n(t, x_1, x_2, \dots, x_n)}\]
\end{definition}

\begin{definition}
\label{def:siseq1ord}
	Un sistema de ecuaciones diferenciales de primer orden lineal es de la
	forma:
	\[\eqsys{
		x'_1 &= a_{11}(t)x_1 + a_{12}(t)x_2 + \dots + a_{1n}(t)x_n + b_1(t) \\
		x'_2 &= a_{21}(t)x_1 + a_{22}(t)x_2 + \dots + a_{2n}(t)x_n + b_2(t) \\
			 &\vdots \\
		x'_n &= a_{n1}(t)x_1 + a_{n2}(t)x_2 + \dots + a_{nn}(t)x_n + b_n(t)}\]
\end{definition}

Podemos expresar un sistema de este tipo en forma matricial como
\[x' = Ax + b\]
donde
\[x = x(t) = \mat{x_1(t) \\ x_2(t) \\ \vdots \\ x_n(t)}, \quad
  A = A(t) = \mat{a_{ij}(t)}^n_{i,j = 1},\quad 
  b = b(t) = \mat{b_1(t) \\ b_2(t) \\ \vdots \\ b_n(t)} \]
y las derivadas e integrales se aplican ``coordenada a coordenada", v.g.
\[x' = \mat{x'_1(t) \\ x'_2(t) \\ \vdots \\ x'_n(t)}
	 = \mat{x'_1 \\ x'_2 \\ \vdots \\ x'_n}, \quad
  A' = \mat{a'_{ij}(t)}^n_{i,j = 1}, \quad
  \int A(s) \dif s = \mat{\int a_{ij} \dif s}^n_{i,j = 1}.\]

\section{Problema del valor inicial}

\begin{definition}
	Sean \(t_0 \in (\alpha, \omega)\) y 
	\(x^0 = \mat{x^0_1, x^0_2, \dots, x^0_n}^t \in \R^n\), el problema del valor
	inicial asociado a~\ref{def:siseq1ord} es de la forma
	\[\eqsys{
		x' = A(t)x + b(t) \\
		x(t_0) = x^0}\]
\end{definition}

Probaremos que el problema de valor inicial tiene solución única en todo el
dominio \((\alpha, \omega)\), para ello representaremos el problema del valor 
inicial como una ecuación integral. Es una estrategia general en análisis 
intentar representar las funciones como integrales debido a su buen
comportamiento. Ser solución del problema del valor inicial equivale a
\[x(t) = x^0 + \int_{t_0}^t A(s)x(s) + b(s) \dif s.\]

Denotaremos por \(C^1(I, \R^n)\) el espacio vectorial de las funciones continuas
del intervalo \(I \subset \R\) en \(\R^n\).

\begin{definition}
	El operador \(T\) es una función
	\begin{align*}
		T : C((\alpha, \omega), \R^n) &\to C((\alpha, \omega), \R^n) \\
		\phi  &\mapsto x^0 + \int_{t_0}^t A(s) \phi(s) + b(s) \dif s
	\end{align*}
\end{definition}

Evidentemente \(T(x) = x\) si demostramos que la función \(T\) tiene un único
punto fijo habremos terminado, hemos definido \(T\) con espacio de
salida y llegada iguales para poder aplicar el teorema del punto fijo de Banach,
aunque necesitaremos que el espacio sea también completo. Sean
\(\alpha < \tilde{\alpha}\) y \(\tilde{\omega} < \omega\), probaremos que 
\(T\) tiene un único punto fijo como operador en 
\([\tilde{\alpha}, \tilde{\omega}]\), haciendo tender 
\(\tilde{\alpha} \to \alpha\) y \(\tilde{\omega} \to \omega\) obtendremos el
resultado deseado.

\begin{theorem}[Punto fijo de Banach]
	Sea \((X, d)\) un espacio métrico completo y \(T : X \to X\) una aplicación
	Lipschitz de constante \(\alpha \in [0, 1]\), contractiva, entonces \(T\)
	tiene un único punto fijo.
\end{theorem}

\begin{corollary}
	Sea \((X, d)\) un espacio métrico completo y \(T : X \to X\) tal que 
	\(T^n = T \circ \stackrel{n}{\cdots} \circ T\) es contractiva para algún
	\(n \in \N\), entonces \(T\) tiene un único punto fijo.
\end{corollary}

\begin{definition}
	Un espacio métrico es completo si toda sucesión de Cauchy es convergente.
\end{definition}

Definimos en \(C([\tilde{\alpha}, \tilde{\omega}], \R^n)\) la distancia 
\[d(f, g) = \norm{f - g}_{\infty} = 
	\max_{t \in [\tilde{\alpha}, \tilde{\omega}]} \norm{f(t) - g(t)}_2 .\]

\end{document}
