\documentclass[../ecuaciones_diferenciales.tex]{subfiles}

\begin{document}

Sean \(x_1, x_2, \dots, x_n \in C^1(\alpha, \omega)\) variables, que son
funciones de clase \(C^1\) sobre un intervalo de \(\R\).

\begin{definition}
	Un sistema de ecuaciones diferenciales de primer orden es un sistema de la
	forma:
	\[\eqsys{
			x'_1 &= f_1(t, x_1, x_2, \dots, x_n) \\
			x'_2 &= f_2(t, x_1, x_2, \dots, x_n) \\
			&\vdots \\
			x'_n &= f_n(t, x_1, x_2, \dots, x_n)}\]
\end{definition}

\begin{definition}
	\label{def:siseq1ord}
	Un sistema de ecuaciones diferenciales lineales de primer orden es de la
	forma:
	\[\eqsys{
			x'_1 &= a_{11}(t)x_1 + a_{12}(t)x_2 + \dots + a_{1n}(t)x_n + b_1(t) \\
			x'_2 &= a_{21}(t)x_1 + a_{22}(t)x_2 + \dots + a_{2n}(t)x_n + b_2(t) \\
			&\vdots \\
			x'_n &= a_{n1}(t)x_1 + a_{n2}(t)x_2 + \dots + a_{nn}(t)x_n + b_n(t)}\]
\end{definition}

donde \(a_{ij}\), \(b_i\) son funciones continuas en un intervalo
\((\alpha, \omega) \subset \R\) para \(i,j = 1, \dots, n\). Podemos expresar un
sistema de este tipo en forma matricial como
\[x' = Ax + b\]
donde
\[x = x(t) = \mat{x_1(t) \\ x_2(t) \\ \vdots \\ x_n(t)}, \quad
	A = A(t) = \mat{a_{ij}(t)}^n_{i,j = 1},\quad
	b = b(t) = \mat{b_1(t) \\ b_2(t) \\ \vdots \\ b_n(t)} \]
y las derivadas e integrales se aplican \textquote{coordenada a coordenada}, 
v.g.
\[x' = \mat{x'_1(t) \\ x'_2(t) \\ \vdots \\ x'_n(t)}
	= \mat{x'_1 \\ x'_2 \\ \vdots \\ x'_n}, \quad
	A' = \mat{a'_{ij}(t)}^n_{i,j = 1}, \quad
	\int A(s) \dif s = \mat{\int a_{ij} \dif s}^n_{i,j = 1}.\]

\section{Problema del valor inicial}

\begin{definition}
	Sean \(t_0 \in (\alpha, \omega)\) y
	\(x^0 = \mat{x^0_1, x^0_2, \dots, x^0_n}^t \in \R^n\), el problema del valor
	inicial asociado a~\ref{def:siseq1ord} es de la forma
	\[\eqsys{
			x' = A(t)x + b(t) \\
			x(t_0) = x^0}\]
\end{definition}

Probaremos que el problema de valor inicial tiene solución única en todo el
dominio \((\alpha, \omega)\), para ello representaremos el problema del valor
inicial como una ecuación integral, es una estrategia general en análisis
intentar representar las funciones como integrales debido a su buen
comportamiento. Ser solución del problema del valor inicial equivale a
\[x(t) = x^0 + \int_{t_0}^t A(s)x(s) + b(s) \dif s.\]

Denotaremos por \(C(I, \R^n)\) el espacio vectorial de las funciones continuas
del intervalo \(I \subset \R\) en \(\R^n\).

A raíz del sistema~\ref{def:siseq1ord} definimos el operador
\begin{align*}
	T : C((\alpha, \omega), \R^n) & \to C((\alpha, \omega), \R^n)                         \\
	\phi                          & \mapsto x^0 + \int_{t_0}^t A(s) \phi(s) + b(s) \dif s
\end{align*}

Evidentemente \(x\) es solución del sistema si y solo si \(T(x) = x\), por lo
que si demostramos que el operador \(T\) tiene un único punto fijo habremos
terminado.

\begin{remark}
	En realidad, \(T(\phi) \in C^1((\alpha, \omega), \R^n)\) para
	\(\phi \in C((\alpha, \omega), \R^n)\); hemos definido \(T\) con espacio de
	llegada \(C((\alpha, \omega), \R^n)\), del que \(C^1((\alpha, \omega), \R^n)\)
	es subespacio vectorial, para poder aplicar el teorema del punto fijo de
	Banach, que requiere espacios de salida y llegada iguales.
\end{remark}

Consideraremos \(\alpha < \tilde{\alpha} < t_0 < \tilde{\omega} < \omega\), y
demostraremos que \(T\) tiene un único punto fijo como operador en
\(C([\tilde{\alpha}, \tilde{\omega}], \R^n)\). Haciendo tender \(\tilde{\alpha}
\to \alpha,\ \tilde{\omega} \to \omega\), tendremos el resultado deseado.

Antes de seguir necesitamos algunos recordatorios:

\begin{definition}
	Un espacio métrico es completo si toda sucesión de Cauchy es convergente.
\end{definition}

\begin{theorem}[Punto fijo de Banach]
	Sean \((X, d)\) un espacio métrico completo y \(T : X \to X\) una aplicación
	contractiva (es decir, Lipschitz con constante \(\alpha \in [0, 1)\)), entonces
	\(T\) tiene un único punto fijo.
\end{theorem}

Este teorema tiene una consecuencia que nos será muy útil.

\begin{corollary}
	Sea \((X, d)\) un espacio métrico completo y \(T : X \to X\) tal que
	\(T^m = T \circ \overset{m}{\cdots} \circ T\) es contractiva para algún
	\(m \in \N\), entonces \(T\) tiene un único punto fijo.
\end{corollary}

Definimos en \(C([\tilde{\alpha}, \tilde{\omega}], \R^n)\) la distancia
\[d(f, g) = \norm{f - g}_{\infty} = \max_{t \in [\tilde{\alpha},
			\tilde{\omega}]} \norm{f(t) - g(t)}_2,\] donde \(\norm{\cdot}_2\) es la
norma euclídea en \(\R^n\). Así \(C([\tilde{\alpha}, \tilde{\omega}], \R^n)\)
es completo, como queríamos:

\begin{proposition}
	\(\left( C([\tilde{\alpha}, \tilde{\omega}], \R^n), d \right)\) es un espacio
	métrico completo.
\end{proposition}

\begin{proof}
	Para comprobar que \(d\) es una distancia, basta ver que
	\[\norm{f}_\infty := \max_{t \in [\tilde{\alpha}, \tilde{\omega}]}
		\norm{f(t)}_2\]
	es una norma (basta usar que \(\norm{\cdot}_2\) es una norma y las
	propiedades del máximo).

	Para ver que es completo, consideramos \((f_n)_n\) una sucesión de Cauchy
	arbitraria en \(C([\tilde{\alpha}, \tilde{\omega}], \R^n)\), lo que quiere
	decir que para cualquier \(\epsilon > 0\) existe un natural \(N_0\) tal que
	si \(n, m \geq N_0\) entonces
	\(\norm{f_n-f_m}_\infty = d(f_n, f_m) < \epsilon\). Fijamos ahora
	\(t \in [\tilde{\alpha}, \tilde{\omega}]\) y consideramos la sucesión
	\((f_n(t))_n \subset \R^n\), que es de Cauchy porque
	\(\norm{f_n(t) - f_m(t)}_2 \leq \norm{f_n - f_m}_\infty\). Usamos la
	completitud de \(\R^n\) para obtener un límite \(f(t) = \lim_n f_n(t)\), lo
	que nos proporciona una función \(f: [\tilde{\alpha}, \tilde{\omega}] \to
	\R^n\) que será la candidata a límite. Lo único que nos queda por ver es que
	\(f \in C([\tilde{\alpha}, \tilde{\omega}], \R^n)\) y \(f_n \overset{d}{\to} f\).

	Dado \(\epsilon > 0\), existe \(N_0 \in \N\) tal que si \(n, m \geq N_0\)
	entonces \(\norm{f_n(t) - f_m(t)}_2 \leq \norm{f_n - f_m}_\infty < \epsilon\)
	para todo \(t \in [\tilde{\alpha}, \tilde{\omega}]\), y haciendo tender
	\(m \to \infty\), se tiene \(\norm{f_n(t) - f(t)}_2 \leq \epsilon\) para todo
	\(t \in [\tilde{\alpha}, \tilde{\omega}]\), de donde
	\(\norm{f_n-f}_\infty \leq \epsilon\).

	Queda así demostrado que \(f_n \to f\) uniformemente; por un lado, esto
	implica que \(f_n \overset{d}{\to} f\) y, por otro, que \(f\) es continua
	(por serlo las \(f_n\)).
\end{proof}

\begin{remark}
	La estrategia de esta demostración para probar la completitud es frecuente en
	espacios de funciones.
\end{remark}

Para demostrar el teorema de existencia y unicidad de la solución, únicamente
nos queda por ver que existe un \(m \in \N\) tal que \(T^m\) es contractiva.

Sean \(f, g \in C([\tilde{\alpha}, \tilde{\omega}], \R^n)\) y \(t \in
[\tilde{\alpha}, \tilde{\omega}]\), entonces
\begin{equation}
	\label{eq:contr}
	\norm{T(f)(t) - T(g)(t)}_2 = \norm{\int_{t_0}^tA(s)(f(s)-g(s))\dif s}_2 \leq
	\int_{t_0}^t \norm{A(s)(f(s)-g(s))}_2\dif s.
\end{equation}

\begin{definition}
	Dada una matriz \(M \in \mathcal{M}_{n \times n}\), definimos su norma como operador como
	\[\norm{M}_{op} = \max_{\norm{x}_2 \leq 1} \norm{Mx}_2 = \sup_{x \neq 0}
		\frac{\norm{Mx}_2}{\norm{x}_2}\]
\end{definition}

\begin{remark}
	Para toda matriz \(M \in \mathcal{M}_{n \times n}\) y todo \(x \in \R^n\) se cumple
	\(\norm{Mx}_2 \leq \norm{M}_{op} \norm{x}_2\).
\end{remark}

\begin{definition}
	Si \(A : X \subset \R \to \mathcal{M}_{n \times n}\), definimos su norma infinito como
	\[\norm{A}_\infty = \max_{x \in X} \norm{A(x)}_{op}\]
\end{definition}

Con esta notación, de~\eqref{eq:contr} se deduce
\begin{align*}
	\norm{T(f)(t) - T(g)(t)}_2 & \leq \int_{t_0}^t
	\norm{A(s)}_{op}\norm{f(s)-g(s)}_2\dif s                               \\
	                           & \leq \int_{t_0}^t \norm{A}_\infty
	\norm{f-g}_\infty\dif s                                                \\
	                           & = \norm{A}_\infty\norm{f-g}_\infty(t-t_0)
\end{align*}

Como \(t, t_0 \in [\tilde{\alpha}, \tilde{\omega}]\), se tiene
\(\norm{T(f)(t)-T(g)(t)}_2 \leq
\norm{A}_\infty\norm{f-g}_2(\tilde{\omega}-\tilde{\alpha})\) y, tomando
supremos, \(\norm{T(f)-T(g)}_\infty \leq \norm{A}_\infty(\tilde{\omega} -
\tilde{\alpha}) \cdot \norm{f-g}_\infty\), es decir, \(T\) es Lipschitz de
constante \(\norm{A}_\infty(\tilde{\omega}-\tilde{\alpha})\). Habríamos
acabado en el caso de que \(\norm{A}_\infty(\tilde{\omega}-\tilde{\alpha}) <
1\); si no, aún nos queda un poco de trabajo:

\begin{lemma}\label{lem:desinfty}
	Para todo \(m \in \N\) se da la desigualdad:
	\[\norm{T^m(f)(t) - T^m(g)(t)}_2 \leq
		\frac{\norm{A}^m_\infty}{m!} \abs{t - t_0}^m \norm{f - g}_\infty.\]
\end{lemma}

\begin{proof}
	El caso base \(m = 1\) está probado; suponemos el resultado cierto para
	\(m\) y para \(m + 1\) se tiene:
	\begin{align*}
		\norm{T^{m + 1}(f)(t) - T^{m + 1}(g)(t)}_2
		 & = \norm{T(T^m(f))(t) - T(T^m (g))(t)}_2 \\
		 & \leq \int_{t_0}^t \norm{A(s)}_{op}
		\norm{T^m(f)(s) - T^m (g)(s)}_2 \dif s,
	\end{align*}
	por la hipótesis de inducción:
	\begin{align*}
		\int_{t_0}^t \norm{A(s)}_{op}
		\norm{T^m(f)(s) - T^m (g)(s)}_2 \dif s
		 & \leq \int_{t_0}^t \norm{A}_\infty
		\frac{\norm{A}^m_\infty}{m!} \abs{s - t_0}^m
		\norm{f - g}_\infty \dif s                                     \\
		 & \leq \frac{\norm{A}^{m + 1}_\infty}{m!} \norm{f - g}_\infty
		\int_{t_0}^t \abs{s - t_0}^m \dif s,
	\end{align*}
	ahora sin más que integrar:
	\begin{align*}
		\norm{T^{m + 1}(f)(t) - T^{m + 1}(g)(t)}_2
		 & \leq \frac{\norm{A}^{m + 1}_\infty}{m!} \norm{f - g}_\infty
		\frac{\abs{t - t_0}^{m + 1}}{m + 1}                                 \\
		 & = \frac{\norm{A}^{m + 1}_\infty}{(m + 1)!} \abs{t - t_0}^{m + 1}
		\norm{f - g}_\infty,
	\end{align*}
	con lo que obtenemos el resultado.
\end{proof}

\begin{remark}
	\(\norm{A(s)(f(s)-g(s))}_2 \leq \norm{A}_\infty \norm{f-g}_\infty\) no es más
	que la propiedad de Lipschitz para \(A\) (no hace falta la linealidad).
\end{remark}

\begin{corollary}
	Se da la desigualdad
	\[\norm{T^m(f) - T^m(g)}_\infty \leq
		\frac{\norm{A}^m_\infty}{m!} (\tilde{\omega} - \tilde{\alpha})^m
		\norm{f - g}_\infty.\]
\end{corollary}

\begin{proof}
	Tomamos supremos en \(t\) en el lema~\ref{lem:desinfty}.
\end{proof}

En particular, existe \(m \in \N\) tal que \(T^m\) es contractiva, y queda así
demostrado el teorema de Picard:

\begin{theorem}[Picard]
	Si \(A(t)\), \(b(t)\) son continuas en \((\alpha, \omega) \subset \R\),
	\(t_0 \in \R\) y \(x^0 \in \R^n\), entonces el problema del valor inicial:
	\[\eqsys{
			x' = A(t)x + b(t) \\
			x(t_0) = x^0}\]
	tiene solución única en \((\alpha, \omega)\).
\end{theorem}

\begin{remark}
	La misma demostración prueba existencia y unicidad para el caso no lineal bajo
	hipótesis de Lipschitzianidad en el \textquote{operador diferencial}.
\end{remark}

Observamos que \(T^m(f_0)\), donde \(f_0\) es un punto inicial cualquiera,
converge exponencialmente rápido al punto fijo de \(T\). Esto nos da un método
numérico para aproximar la solución del PVI escogiendo un punto inicial simple
(constante). También nos proporciona el desarrollo en serie (analítico) de la
solución del PVI, este método se conoce como método de las iteraciones de
Picard.

\begin{example}
	Consideramos el PVI:
	\[\eqsys{x' = ax \quad \text{(ecuación escalar)} \\
			x(0) = x_0}\]

	Iniciamos el proceso con la solución constante \(x_0\),
	\[x_1 = T(x_0) = x_0 + \int_0^t a x_0 \dif s = x_0 + x_0 a t,\]
	reiterando tenemos que
	\[x_n = T(x_{n - 1}) = x_0 \sum_{k = 0}^n \frac{(at)^k}{k!},\] cuando \(n\)
	tiende a infinito nos queda \(x_\infty := \lim_n x_n = x_0 e^{at}\).
\end{example}

\begin{remark}
	Este procedimiento también vale, con las modificaciones obvias, para el caso
	matricial, a diferencia de la separación de variables que ya vimos.
\end{remark}

\section{Estructura del espacio de soluciones}

\subsection{Caso homogéneo}

Consideramos el sistema homogéneo \(x' = A(t) x\) y definimos la función:
\begin{align*}
	T : C^1((\alpha, \omega), \R^n) & \to C               \\
	x                               & \mapsto x' - A(t)x,
\end{align*}
claramente \(x\) es solución de la ecuación si y solo si \(x \in \ker(T)\).
Puesto que \(T\) es lineal el conjunto de soluciones es un espacio vectorial,
más propiamente un subespacio vectorial de \(C^1((\alpha, \omega), \R^n)\).

\begin{theorem}
	El espacio de soluciones de un sistema de ecuaciones lineales homogéneas es
	un subespacio vectorial de \(C^1((\alpha, \omega), \R^n)\) de dimensión
	\(n\). Además, si \(\phi_j\) es la solución del problema de valor inicial
	\[\eqsys{
			x' = A(t)x \\
			x(t_0) = e_j = (0, \dots, \underset{j}{1}, \dots, 0)^T}\]
	siendo \(t_0 \in (\alpha, \omega)\) arbitrario, tenemos que
	\(\set{\phi_1, \dots, \phi_n}\) es base del susodicho espacio de soluciones.
\end{theorem}

\begin{proof}
	Demostramos primero que los \(\phi_j\) son linealmente independientes. Si
	\(\alpha_1, \dots, \alpha_n\) son tales que \(\sum_{j = 1}^n \alpha_j \phi_j
	\equiv 0\), evaluando en \(t_0\) se obtiene
	\[\sum_{j = 1}^n \alpha_j e_j = 0 \implies \alpha_j = 0,\]
	para todo \(j = 1, \dots, n\) como queríamos, por ser \(e_1, \dots,
	e_n\) base. Demostramos ahora que es sistema de generadores. Sea \(y\) solución
	de la ecuación lineal y sea \(y(t_0) = (\alpha_1, \dots, \alpha_n)^t\),
	definimos \(z = \sum_{j = 1}^n \alpha_j \phi_j\) y observamos que tanto
	\(y\) como \(z\) son soluciones de
	\[\eqsys{
			x' = A(t)x \\
			x(t_0) = (\alpha_1, \dots, \alpha_n)^t}\]
	de donde deducimos que \(y = z\) en virtud del Teorema de Picard.
\end{proof}

\begin{definition}
	Llamaremos matriz fundamental del sistema \(x' = A(t)x\) y la denotaremos
	\(\Phi(t)\) a cualquier matriz cuyas columnas formen base del espacio de
	soluciones del sistema.
\end{definition}

Con las definiciones anteriores tenemos que
\(\Phi(t) := (\phi_1, \phi_2, \dots, \phi_n)\) es una matriz fundamental. Por la
propia definición tenemos que si \(\Phi\) es una matriz fundamental de un
sistema de ecuaciones lineales homogéneas entonces la solución general del
sistema es \(\Phi(t)c\) con \(c \in \R^n\):
\[x(t) = \Phi(t) c = (\phi_1, \dots, \phi_n) \mat{c_1 \\ \vdots \\ c_n} =
	c_1\phi_1 + \cdots + c_n\phi_n.\]

\begin{corollary}
	Sea \(\Phi(t)\) una matriz cuyas columnas son solución del sistema
	\(x' = A(t)x\), entonces son equivalentes:

	\begin{enumerate}[i)]
		\item \(\Phi(t)\) es matriz fundamental.

		\item \(\det(\Phi(t)) \neq 0 \ \forall t \in (\alpha, \omega)\).

		\item \(\exists t_0 \in (\alpha, \omega) \ \det(\Phi(t_0)) \neq 0\).
	\end{enumerate}
\end{corollary}

Las columnas de \(\Phi(t)\) son soluciones de \(x' = A(t)x\) si y solo si
\(\Phi(t)\) es solución de la ecuación matricial \(X' = A(t)X\), donde
\begin{align*}
	X : (\alpha, \omega) & \to \mathcal{M}_{n \times n} \\
	t                    & \mapsto X(t)
\end{align*}

Asimismo, que \(\Phi(t)\) sea matriz fundamental equivale a que sea solución de
un problema de valor inicial de la forma:
\[\eqsys{
		X' = A(t)X \\
		X(t_0) = X^0}\]
donde ambas ecuaciones son matriciales y \(\det(X_0) \neq 0\). Notamos que la
matriz fundamental \(\Phi(t) = (\phi_1, \dots, \phi_n)\) que definimos
anteriormente es la única solución del problema de valor inicial:
\[\eqsys{
		X' = A(t)X \\
		X(0) = \mathit{Id}}\]

\subsection{Caso no homogéneo}

La misma demostración del caso escalar prueba que la solución general de
\(x' = A(t)x + b(t)\) es \(x_p(t) + x_h(t)\), donde \(x_p(t)\) es una solución
particular arbitraria del sistema y \(x_h(t)\) es la solución general del
sistema homogéneo asociado \(x' = A(t)x\). Por lo visto arriba,
\(x_h(t) = \Phi(t)c\) con \(c \in \R^n\), y como veremos más adelante existen
diversos métodos para obtener \(x_p(t)\).

\section{Método de variación de constantes}

Un método para obtener una solución particular \(x_p(t)\) de un sistema
\(x' = A(t)x + b(t)\) es la variación de constantes, que como veremos es muy
similar al al caso de una variable. Sabemos que la solución general de la
ecuación homogénea asociada es \(x_h(t) = \Phi(t)c\), conjeturamos la existencia
de \(x_p(t) = \Phi(t) c(t)\):
\[x'_p(t) = \Phi'(t) c(t) + \Phi(t) c'(t) = A(t )\Phi(t) c(t) + \Phi(t) c'(t).\]

Además, puesto que es solución del sistema homogéneo
\[x'_p(t) = A(t)x_p(t) + b(t) = [A(t) \Phi(t) c(t)] + b(t),\]

igualando ambas expresiones
\[c'(t) = \Phi^{-1}(t) b(t).\]

Basta por tanto elegir \(c(t) = \int_{t_0}^t \Phi^{-1}(s) b(s) \dif s\), hemos
demostrado el siguiente teorema:

\begin{theorem}
	Sea \(\Phi(t)\) una matriz fundamental del sistema \(x' = A(t)x\), entonces
	\[x_p(t) = \Phi(t) \int_{t_0}^t \Phi^{-1}(s) b(s) \dif s\]
	con \(t_0 \in (\alpha, \omega)\) es una solución particular de
	\(x' = A(t)x + b(t)\) que satisface \(x_p(t_0) = 0 \in \R^n\).
\end{theorem}

\end{document}
