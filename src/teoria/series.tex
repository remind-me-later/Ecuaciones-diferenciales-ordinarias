\documentclass[../ecuaciones_diferenciales.tex]{subfiles}

\begin{document}

En este capítulo presentamos un método para obtener soluciones a ecuaciones
diferenciales lineales en forma de series de potencias. Comenzamos recordando
algunos resultados de análisis real que nos serán útiles.

\section{Repaso de series de potencias}

\begin{definition}[Serie de potencias]
	Una serie de funciones \(\sum f_n\) es una serie de potencias alrededor de
	\(t_0\) si es una serie de la forma
	\[\sum_{n = 0}^\infty a_n (t - t_0)^n, \quad a_n \in \R.\]
\end{definition}

Puesto que haremos uso de series de potencias no tenemos otra opción 
que restringirnos a soluciones \emph{analíticas}.

\begin{definition}[Función analítica]
	Una función \(f\) es analítica en un intervalo abierto \(I\) si para cada
	\(t_0 \in I\) se puede expresar como una serie de potencias:
	\[f(t) = \sum_{n = 0}^\infty a_n (t - t_0)^n\]
	convergente a \(f(t)\) para todo \(t\) en un entorno de \(t_0\).
\end{definition}

Necesitaremos una forma de calcular el conjunto donde estas series convergen,
que como veremos es un intervalo centrado en \(t_0\) o todo \(\R\), para ello 
recordamos algunos resultados sobre convergencia.

\begin{definition}[Radio de convergencia]
	Sea \(\sum a_n (t - t_0)^n\) una serie de potencias. Si la sucesión 
	\(\abs{a_n}^{1/n}\) es acotada definimos 
	\(\rho := \lim\sup\abs{a_n}^{1/n}\), si esta sucesión no es acotada 
	\(\rho = \infty\). 

	Definimos el \emph{radio de convergencia} de \(\sum a_n (t - t_0)^n\) como
	\[R := 
		\begin{cases}
			0 & \text{si} \quad \rho = \infty \\
			1/\rho & \text{si} \quad 0 < \rho < \infty \\
			\infty & \text{si} \quad \rho = 0
		\end{cases}
	\]
\end{definition}

Con esto podemos definir el término \textquote{intervalo de convergencia}.

\begin{definition}[Intervalo de convergencia]
	Sea \(\sum a_n (t - t_0)^n\) una serie de potencias y \(R\) su radio de
	convergencia. El intervalo de convergencia de la serie es \((-R, R)\).
\end{definition}

El teorema que enunciamos ahora justificará el nombre \textquote{radio de
convergencia}.

\begin{theorem}[Cauchy-Hadamard]
	Si \(R\) es el radio de convergencia de la serie de potencias 
	\(\sum a_n (t - t_0)^n\), entonces la serie es absolutamente convergente si 
	\(\abs{t} < R\) y diverge si \(\abs{t} > R\).
\end{theorem}

\begin{remark}
	Notamos que el teorema precedente no dice nada acerca de la convergencia de
	la serie cuando \(\abs{t} = R\), en este caso tanto la convergencia como la
	divergencia son posibles y debemos analizarlo por separado.
\end{remark}

La forma de calcular el radio de convergencia dada en el teorema de
Cauchy-Hadamard puede ser tediosa en algunos casos, por lo que mostramos una
forma alternativa de computar el radio.

\begin{lemma}
	Sea \(\sum a_n (t - t_0)^n\) una serie de potencias, su radio de
	convergencia viene dado por
	\[\lim\abs{\frac{a_n}{a_{n + 1}}},\]
	si este límite existe o es infinito.
\end{lemma}

Para finalizar recordamos también un par de teoremas que nos permitirán
diferenciar e integrar cómodamente este tipo de series, ambos consecuencia casi
inmediata del siguiente lema.

\begin{lemma}
	Sea \(R\) el radio de convergencia de una serie de potencias y \(V\) un
	intervalo cerrado y acotado contenido en el intervalo de convergencia 
	\((-R, R)\), entonces, la serie de potencias converge uniformemente en
	\(V\).
\end{lemma}

Puesto que la serie de potencias converge uniformemente en cualquier compacto
contenido en el intervalo de convergencia tenemos que su límite es continuo
dentro de este intervalo, y no solo eso, además es integrable término a
término y derivable término a término.

\begin{corollary}
	El límite de una serie de potencias es continuo en su intervalo de
	convergencia.
\end{corollary}

\begin{theorem}[Integración]
	Una serie de potencias puede ser integrada término a término en cualquier
	intervalo cerrado y acotado contenido en su intervalo de convergencia. O lo
	que es lo mismo:
	\[\int_a^b f = \int_a^b \sum_{n = 0}^\infty a_n (t - t_0)^n
		= \sum_{n = 0}^\infty \int_a^b a_n (t - t_0)^n
		= c + \sum_{n = 0}^\infty a_n \frac{(t - t_0)^{n + 1}}{n + 1}.\]
\end{theorem}

En el caso de la derivación presentamos un teorema más fuerte que el de la
derivación de series de funciones uniformemente convergentes.

\begin{theorem}[Diferenciación]
	Una serie de potencias puede ser diferenciada término a término en su
	intervalo de convergencia. Es más, si
	\[f(x) = \sum_{n = 0}^\infty a_n (t - t_0)^n, \quad \text{entonces} \quad
		f'(x) = \sum_{n = 0}^\infty a_n n (t - t_0)^{n - 1},\]
	para \(\abs{x} < R\) y ambas series tienen el mismo radio de convergencia.
\end{theorem}

Remitimos al lector que quiera demostraciones detalladas de estos
teoremas al capítulo 9.4 de \cite{introduction_real_analysis}.

\section{Aplicación a la resolución de ecuaciones diferenciales}

Después de este pequeño repaso podemos presentar el método en cuestión, para
ello damos primero un ejemplo que muestra inmediatamente su utilidad.
\begin{example}
	Expresar las soluciones a la ecuación
	\begin{equation} \label{eq:eqpot}
		x''-2tx'-2x=0
	\end{equation}
	como series de potencias.
\end{example}

\begin{solution}
	Buscamos soluciones analíticas con radio de convergencia \(\geq \rho\) y
	estamos interesados en el rango de tiempos \(t\) tal que \(|t-t_0| <
	\rho\). En este ejemplo tomamos \(t_0 = 0\), por lo que las soluciones serán
	de la forma
	\begin{equation}
		\label{eq:solpot}
		x(t) = \sum_{n=0}^\infty a_nt^n, \quad |t| < \rho
	\end{equation}

	Nuestro objetivo se reduce, entonces, a dar \textquote{ecuaciones} que 
	caractericen la sucesión \((a_n)_n\), de forma que, al resolverlas, 
	obtengamos la función \(x(t)\).

	Derivando en~\eqref{eq:solpot}, llegamos a las series de potencias
	\begin{align*}
		x'(t)  & = \sum_{n=1}^\infty na_nt^{n-1}
		= \sum_{n=0}^\infty (n+1)a_{n+1}t^n,          \\
		x''(t) & = \sum_{n=2}^\infty n(n-1)a_nt^{n-2}
		= \sum_{n=0}^\infty (n+2)(n+1)a_{n+2}t^n,
	\end{align*}
	donde ambas series de potencias tienen radio de convergencia \(\geq \rho\).

	Teniendo esto en cuenta y volviendo a~\eqref{eq:eqpot}, tenemos
	\[x''-2tx'-2x=0 \leftrightsquigarrow \sum_{n=0}^\infty (n+2)(n+1)a_{n+2}t^n -
		2t \sum_{n=0}^\infty (n+1)a_{n+1}t^n - 2 \sum_{n=0}^\infty a_nt^n = 0\]
	o, equivalentemente,
	\[\sum_{n=0}^\infty (n+2)(n+1)a_{n+2}t^n - \sum_{n=0}^\infty
		2(n+1)a_{n+1}t^{n+1} - \sum_{n=0}^\infty 2a_nt^n = 0.\]
	Escribimos esto como una única serie de potencias:
	\[\underbrace{2a_2-2a_0}_{n=0} +
		\sum_{n=1}^\infty[(n+1)(n+2)a_{n+2}-2na_n-2a_n]t^n = 0\]
	que, como es idénticamente nula, tiene todos sus coeficientes iguales a 0:
	\begin{equation}
		\label{eq:seqpot}
		\begin{cases}
			a_2 = a_0                                 \\
			(n+1)(n+2)a_{n+2} = 2(n+1)a_n, & n \geq 1
		\end{cases} \iff
		a_{n+2} = \frac{2}{n+2} a_n, \qquad n \geq 0
	\end{equation}

	En resumen,~\eqref{eq:solpot} es solución de ~\eqref{eq:eqpot} si y sólo si
	\((a_n)_n\) verifica estas condiciones. Una solución queda unívocamente
	determinada al fijar \(x(0)\) y \(x'(0)\), que en este caso corresponden,
	respectivamente, a \(a_0\) y \(a_1\). Como ya sabemos, una base del espacio de
	soluciones de~\eqref{eq:eqpot} se obtiene al resolver~\eqref{eq:seqpot} para los
	casos
	\[\eqsys{a_0 = 1 \\ a_1 = 0}
		\quad \text{y} \quad
		\eqsys{a_0 = 0 \\ a_1 = 1}\]

	\begin{itemize}
		\item Caso \(a_0 = 1,\ a_1 = 0\):
		      \[a_2 = 1, \quad a_3 = 0, \quad a_4 = \frac{1}{2}, \quad a_5 = 0, \quad a_6 =
			      \frac{1}{3}a_4 = \frac{1}{3!}, \quad a_7 = 0, \quad a_8 = \frac{1}{4}a_6 =
			      \frac{1}{4!}, \quad \cdots\]
		      Resulta evidente que
		      \[a_{2k} = \frac{1}{k!} \quad \text{y} \quad a_{2k+1} = 0,\]
		      y es inmediato probarlo por inducción.

		      En este caso la solución es
		      \[x(t) = \sum_{k=0}^\infty \frac{t^{2k}}{k!} = e^{t^2}, \qquad \rho =
			      +\infty\]

		\item Caso \(a_0 = 0,\ a_1 = 1\):
		      \[a_2 = 0, \quad a_3 = \frac{2}{3}, \quad a_4 = 0, \quad a_5 = \frac{2}{5}a_3
			      = \frac{2^2}{3 \cdot 5}, \quad a_6 = 0, \quad a_7 = \frac{2}{7}a_5 =
			      \frac{2^3}{3 \cdot 5 \cdot 7}, \quad \cdots\]
		      La fórmula general es ahora
		      \[a_{2k} = 0 \quad \text{y} \quad a_{2k+1} = \frac{2^k}{\prod_{i=1}^k
				      (2i+1)},\]
		      e igual que antes es inmediato probarlo por inducción.

		      En este caso, \(\sum_{n=0}^\infty a_nt^n\) no es una función
		      elemental, aunque es analítica con radio de
		      convergencia \(\rho = +\infty\).
	\end{itemize}
\end{solution}

Para terminar con el capítulo, recogemos las ideas generales en un teorema:
\begin{theorem}
	Supongamos que \(Q(t)\), \(R(t)\) y \(P(t)\) se pueden escribir como series
	de potencias alrededor de \(t_0\) con radio de convergencia \(\rho\):
	\[Q(t) = \sum_{n=0}^\infty q_n(t-t_0)^n, \quad R(t) = \sum_{n=0}^\infty
		r_n(t-t_0)^n, \quad P(t) = \sum_{n=0}^\infty p_n(t-t_0)^n,
		\qquad |t-t_0| < \rho\]
	Entonces, \emph{todas} las soluciones de la ecuación
	\begin{equation} \label{eq:thmpot}
		x'' + Q(t)x' + R(t)x = P(t)
	\end{equation}
	se pueden escribir también como series de potencias alrededor de
	\(t_0\) con radio de convergencia \(\rho\), es decir,
	\[x(t) = \sum_{n=0}^\infty a_n(t-t_0)^n, \qquad |t-t_0| < \rho.\]
	Además, en este caso, para obtener los coeficientes \(a_n\) se puede razonar
	como en el ejemplo: se asume \(x(t) = \sum_{n=0}^\infty a_n(t-t_0)^n\), se
	escribe~\eqref{eq:thmpot} como serie de potencias y se igualan sus
	coeficientes a 0. Esto da unas ecuaciones que caracterizan la sucesión
	\((a_n)_n\).
\end{theorem}

\begin{remark}
	Para expresar el producto de dos series de potencias como serie de potencias
	se usa el \emph{producto de Cauchy} o \emph{convolución}:
	\[\left( \sum_{n=0}^\infty A_n (t-t_0)^n \right) \cdot
		\left( \sum_{n=0}^\infty B_n (t-t_0)^n \right) =
		\sum_{n=0}^\infty C_n (t-t_0)^n,\]
	siendo
	\[C_n = \sum_{k=0}^n A_kB_{n-k}.\]
\end{remark}

\end{document}
