\documentclass[../main.tex]{subfiles}

\begin{document}
Hasta ahora nos hemos preocupado de \emph{resolver} los sistemas de ecuaciones
diferenciales lineales. Ahora queremos entender el comportamiento
\emph{cualitativo} de sus soluciones.

Haciendo una analogía con el análisis real, un planteamiento cuantitativo es
tener una fórmula explícita para una función y uno cualitativo es dar la forma
global de su gráfica (crecimiento, convexidad, asíntotas...)

Si sabemos resolver explícitamente los sistemas, ¿por qué interesa hacer un
análisis cualitativo?

\begin{enumerate}[1)]
\item En muchos casos, nos da información relevante del modelo estudiado. Por
  ejemplo, es importante saber si una enfermedad desaparece o se hace endémica,
  si una población oscila, se estabiliza o desaparece, si un tumor crece sin
  control o su crecimiento se termina estabilizando, si las oscilaciones de un
  puente acabarán pasando el umbral de rotura, etc.
\item En el caso no lineal, esto es esencialmente lo único que se puede decir
  (aparte de dar aproximaciones numéricas de la solución).
\end{enumerate}

Nos centraremos exclusivamente en \emph{sistemas planos} (matrices \(2 \times 2\))
y en el caso homogéneo de coeficientes constantes, es decir, en sistemas de la
forma
\[x' = Ax, \qquad A = \mat{a_{11} & a_{12} \\ a_{21} & a_{22}}.\]
Estamos interesados en pintar las trayectorias u órbitas:

\begin{definition}
  Dada una solución \((x_1(t), x_2(t))\) de un sistema \(x' = Ax\), su
  \emph{trayectoria} u \emph{órbita} es el conjunto
  \(\{(x_1(t), x_2(t)) : t \in \R\} \subset \R^2\).
\end{definition}

\begin{definition}
  El conjunto de todas las trayectorias de \(x' = Ax\) se llama \emph{diagrama
    de fases} del sistema.
\end{definition}

\begin{example}
  Dibujar el diagrama de fases del sistema \(x' = Ax\) siendo
  \(A = \smat{0 & 1 \\ -1 & 0}\).

  Como ya vimos, la solución general viene dada por
  \[\begin{cases}
      x_1(t) = k_1 \cos t + k_2 \sin t \\
      x_2(t) = - k_1 \sin t + k_2 \cos t
    \end{cases}\]

  Nuestro objetivo es eliminar \(t\) de estas ecuaciones para obtener la
  relación entre \(x_1\) y \(x_2\) que define la trayectoria.

  La solución se puede escribir como

  \[\mat{x_1 \\ x_2} = e^{At}\mat{k_1 \\ k_2} =
    \underbrace{\mat{\cos t & \sin t \\ -\sin t & \cos t}}_{\text{matriz de
        rotación}} \mat{k_1 \\ k_2},\]

  así que, intuitivamente, la trayectoria es una circunferencia que pasa por
  \(\smat{k_1 \\ k_2}\). Analíticamente,
  \begin{align*}
    x_1^2 + x_2^2 &= (k_1^2\cos^2t + k_2^2\sin^2t + 2k_1k_2\cos t \sin t) +
                    (k_1^2\sin^2t + k_2^2\cos^2t - 2k_1k_2\cos t \sin t) \\
    &= k_1^2(\cos^2 t + \sin^2 t) + k_2^2(\cos^2t + \sin^2t) = k_1^2+k_2^2.
  \end{align*}
  Efectivamente, la trayectoria que pasa por \(\smat{k_1 \\ k_2}\) es la
  circunferencia centrada en el origen de radio \(\sqrt{k_1^2+k_2^2}\).
\end{example}

\begin{theorem}
  Las trayectorias no se cortan.
  \begin{proof}
    Supongamos que las trayectorias asociadas a las soluciones \(x(t), y(t)\) de
    \(x' = Ax\) se cortan o, en otras palabras, que existen \(t_1, t_2\) tales
    que \(x(t_1) = y(t_2)\). Lo que ocurre es que \(y(t) = x(t + (t_1-t_2))\),
    luego la trayectoria que definen ambas soluciones resulta ser la misma.

    Para ver esto, definimos \(z(t) = x(t + (t_1-t_2))\). Se cumple que \(z' =
    Az\) y \(z(t_2) = x(t_1) = y(t_2)\), luego, en virtud del teorema de
    existencia y unicidad, debe ser \(z(t) = y(t)\).
  \end{proof}
\end{theorem}

\begin{remark}
  Ya vimos que las gráficas de \(x(t), y(t)\) no se cortan (como subconjuntos de
  \(\R^3\)). Este resultado, más fuerte, quiere decir que no se cortan \emph{sus
    proyecciones}, y se cumple porque \(A\) tiene coeficientes constantes (en
  caso contrario, no sería necesariamente cierto que \(z' = Az\)).
\end{remark}

\section{Interpretación geométrica de las trayectorias}

Una matriz \(A\), o más precisamente su aplicación lineal asociada \(A : \R^2
\to \R^2,\ x \mapsto Ax\), es un \emph{campo vectorial}. Una solución del
sistema \(x' = Ax\) no es más que una curva \(t \mapsto x(t)\) tal que su
velocidad \(x'(t)\) coincide con el valor del campo en \(x(t)\), es decir, una
\emph{curva integral} del campo.

\begin{example}
  Sea la matriz \(A = \smat{0 & 1 \\ -1 & 0}\). Esbozar el campo vectorial que
  define.

  La matriz \(A\) actúa sobre un vector genérico \(\smat{x_1 \\ x_2}\) como
  \(\smat{x_1 \\ x_2} \mapsto A\smat{x_1 \\ x_2} = \smat{x_2 \\ -x_1}\).
  Observamos que \(\langle x, Ax \rangle = 0\), es decir, la velocidad es
  ortogonal a la posición en todo punto, y además \(\|x\| = \|Ax\|\), es decir,
  \(A\) no cambia la norma de los vectores. El campo vectorial que define esta
  matriz es, entonces,

  \begin{figure}[ht]
    \centering
    \begin{tikzpicture}[scale=0.6]
      \draw[->] (-3.5,0) -- (3.5,0) node[above right] {\(x_1\)};
      \draw[->] (0,-3.5) -- (0,3.5) node[above right] {\(x_2\)};
      \foreach \r in {1, 2, 3}
      \foreach \a in {0,...,7}
      \draw[->] ({\r*cos(45*\a)}, {\r*sin(45*\a)}) --
      ({\r*cos(45*\a)+\r/2*sin(45*\a)}, {\r*sin(45*\a)-\r/2*cos(45*\a)});
    \end{tikzpicture}
  \end{figure}
\end{example}

Nos centramos ahora en esbozar el diagrama de fases de un sistema plano general
\(x' = Ax\), siendo \(B = P^{-1}AP\) su forma de Jordan. Las formas de Jordan
posibles son

\begin{enumerate}[a)]
\item \(B = \smat{\lambda_1 & \\ & \lambda_2},\ \lambda_1 \neq \lambda_2\)
\item \(B = \smat{\lambda & \\ & \lambda}\)
\item \(B = \smat{\lambda & 1 \\ & \lambda}\)
\item \(B = \smat{a & b \\ -b & a},\ b>0\)
\end{enumerate}

Haremos los diagramas de fase para las formas de Jordan, y luego desharemos el
cambio de base para obtener el diagrama final.
\end{document}