\documentclass[../ecuaciones_diferenciales.tex]{subfiles}

\begin{document}

Pasamos ahora a desarrollar la teoría fundamental de ecuaciones diferenciales,
comenzando por los resultados de existencia y unicidad de soluciones y los intervalos en
los que están definidas.

\begin{definition}[Ecuación general no lineal]
    Llamamos ecuación general no lineal a una ecuación de la forma 
    \[x' = f(t, x)\]
    donde \(f\) es continua en un abierto \(D \subset \mathbb{R}^2\).
\end{definition}

Veremos que demostrar la existencia y unicidad de soluciones no es tan sencillo como en el caso lineal, ya
que no podremos hacerlo de manera constructiva, ni podremos en general dar fórmulas explícitas en términos
de funciones elementales. Por ejemplo la ecuación:
\[x' = t + x^2\]
no tiene solución elemental. Veremos luego que se trata de un caso particular de \emph{ecuación de Ricatti}.

\begin{definition}[Ecuación de Ricatti]
    Una ecuación de Ricatti es una ecuación de la forma
    \[x' = p(t)x^2 + q(t)x + r(t)\]
    donde \(p, q, r\) son funciones continuas en un abierto \(D \subset \mathbb{R}\).
\end{definition}

Liouville demostró que en general estas ecuaciones no se pueden resolver en términos de funciones elementales.
Aún así son importantes ya que se puede dmeostrar que toda ecuación diferencial lineal homogénea de orden se 
puede reducir a una ecuación de Ricatti.

Con esto queda claro que las ecuaciones diferenciales no lineales resolubles en términos de funciones elementales
conforman un subconjunto muy pequeño de todas las ecuaciones diferenciales no lineales. Tendremos pues que 
abordar la cuestión de existencia y unicidad mediante medios indirectos. Como en la parte anterior lo haremos
prestando atención al problema de valor inicial.

\section{Existencia y unicidad de soluciones}

Sea \(f(t, x)\) definida en un abierto \(D \subset \mathbb{R}^2\). Diremos que \(x(t)\) es \emph{solución} de la
ecuación \(x' = f(t, x)\) en el intervalo abierto \(I\) si \(x\) tiene derivada continua en \(I\) y se verifican:

\begin{multicols}{2}
    \begin{enumerate}[(i)]
        \item \(\forall t \in I, \, (t, x(t)) \in D\)
        \item \(\forall t \in I, \, x'(t) = f(t, x(t))\)
    \end{enumerate}    
\end{multicols}

Observamos que la definición de la solución \(x(t)\) depende del intervalo \(I\) que a su vez
depende del abierto \(D\) en el que se plantea la ecuación.

Abordamos la cuestión de existencia de soluciones centrando la atención en el problema de valor inicial 
asociado a la ecuación \(x' = f(t, x)\):
\[
    \begin{cases}
        x' = f(t, x) \\
        x(t_0) = x_0; \, (t_0, x_0) \in D
    \end{cases}
\]

Comenzamos tranformando este problema en uno más manejable analiticamente: sea \(x(t)\) solución 
del PVI en un intervalo \(I\) que contiene a \(t_0\), entonces, por definición, \(\forall t \in I, \, x'(t) = f(x, x(t))\).
Integrando esta expresión entre \(t_0\) y \(t\) obtenemos:
\begin{equation}
    \label{eq:ex_unicidad:teoria:1}
    x(t) = x_0 + \int_{t_0}^t f(s, x(s)) \, ds
\end{equation}

Por el teorema fundamental del cálculo obtenemos el recíproco: si \(x(t)\) es una función continua en \(I\) que verifica~\ref{eq:ex_unicidad:teoria:1}
basta derivar para obtener que \(x'(t) = f(t, x(t))\). Además evaluando~\ref{eq:ex_unicidad:teoria:1} en \(t_0\) comprobamos
que cumple la condición inicial con lo que es solución del PVI.\@ Hemos encontrado así una \emph{ecuación integral equivalente} al PVI, la 
importancia de esta forma es que se presta mejor a ser aprozimada mediante métodos iterativos. 

Podemos utilizar el método de las \emph{iteraciones de Picard}, que ya vimos en~\ref{cor:picard_iter_lin}, para aproximar la solución del PVI.\@

\begin{definition}[Iteraciones de Picard, versión integral]
Sea \(f(t, x)\) una función continua en un abierto \(D \subset \mathbb{R}^2\), definimos la sucesión de funciones
\begin{equation} \label{eq:ex_unicidad:teoria:2}
    \begin{cases}
        x_0(t) = x_0 \\
        x_{n+1}(t) = x_0 + \int_{t_0}^t f(s, x_n(s)) \, ds
    \end{cases}
\end{equation}
a las que llamaremos iteraciones de Picard.
\end{definition}

Obtenemos así una sucesión de funciones \(\{x_n(t)\}\) que podremos demostrar que converge a una función
\(x(t)\) en un intervalo \(I\) para la que además será legítimo el paso al 
límite \(n \to \infty\) en~\ref{eq:ex_unicidad:teoria:2}. 

La fórmula integral también nos permitirá usar una acotación que será fundamental en todo lo que sigue.

\begin{lemma}[Lema de Gronwall]\label{lemma:ex_unicidad:teoria:gronwall}
    Sea \(u : [a, b] \to \R\) continua y no negativa tal que
    \begin{equation}
        u(t) \leq C + \abs{\int_{t_0}^t K u(s) \, ds}
    \end{equation}
    con \(t, t_0 \in [a, b]\) y \(C \geq 0\), \(K \geq 0\) constantes.
    Entonces 
    \begin{equation}\label{eq:ex_unicidad:teoria:7}
        u(t) \leq C e^{K \abs{t - t_0}}
    \end{equation}
    para todo \(t \in [a, b]\).
\end{lemma}

\begin{proof}
    La hacemos para \(t > t_0\). Definimos 
    \[
        U(t) = C + \int_{t_0}^t K u(s) \, ds
    \]

    Se tiene \(u(t) \leq U(t)\) y \(U'(t) = K u(t) \leq K U(t)\), es decir, 
    \(U'(t) - K U(t) \leq 0\). Multiplicando esta desigualdad por \(e^{-Kt}\) obtenemos
    \[
        e^{-Kt} (U'(t) - K U(t)) = \frac{d}{dt} (e^{-Kt} U(t)) \leq 0, \, t \in [t_0, b]
    \]
    con esto
    \[
        e^{-Kt} U(t) \leq e^{-Kt_0} U(t_0) = C e^{-Kt_0}
    \]
    y por tanto \(U(t) \leq C e^{K(t - t_0)}\) lo que nos da~\ref{eq:ex_unicidad:teoria:7}, ya que \(u(t) \leq U(t)\).
\end{proof}

\subsection{Existencia y unicidad local con derivada continua}

Comenzamos demostrando una versión débil del teorema cuando \(\partial f / \partial x\) es continua.
La clave consistirá en usar la siguiente acotación.

\begin{lemma}~\label{lemma:ex_unicidad:teoria:acotacion}
    Sea \(f(t, x)\) continua en un abierto \(D \subset \mathbb{R}^2\). Supongamos además que \(f\) tiene derivada parcial respecto a \(x\) en todo punto de \(D\)
    y que \(\partial f / \partial x\) es también continua en \(D\). Existe una constante \(K\) tal que para todo rectángulo \(R\) contenido en \(D\)
    \begin{equation}
        \abs{f(t, x_1) - f(t, x_2)} \leq K \abs{x_1 - x_2} 
    \end{equation}
    para todo par de puntos \((t, x_1)\) y \((t, x_2)\) contenidos en \(R\) con la misma abscisa. Además
    \[
        K = \max_{(t, x) \in R} \abs{\frac{\partial f}{\partial x}(t, x)}
    \]
\end{lemma}

\begin{proof}
    Sean \(a > 0\) y \(b > 0\) tal que el rectángulo
    \[
        R = \{(t, x) : \abs{t - t_0} \leq a, \abs{x - x_0} \leq b\}
    \]
    está contenido en \(D\). Como \(R\) es cerrado y acotado y \(f\) y \(\partial f / \partial x\)
    son, por hipótesis, continuas en \(D\) podemos definir
    \[
        K = \max_{(t, x) \in R} \abs{\frac{\partial f}{\partial x}(t, x)}
    \]
    Ahora, por el teorema del valor medio si \((t, x_1)\) y \((t, x_2)\) son dos
    puntos de \(R\) con la misma abscisa
    \[
        \abs{f(t, x_1) - f(t, x_2)} = \abs{\frac{\partial f}{\partial x}(t, \xi)} \abs{x_1 - x_2}
    \]
    con \(\xi\) entre \(x_1\) y \(x_2\). De esto se concluye directamente el resultado.
\end{proof}

Estamos ahora en condiciones de enunciar 
y demostrar el teorema de existencia y unicidad de soluciones.

\begin{theorem}[Existencia y unicidad local]\label{theorem:ex_unicidad:teoria:teorema}
    Sea \(f(t, x)\) continua en un abierto \(D \subset \mathbb{R}^2\) y dentro de ese abierto
    consideramos el intervalo \((t_0, x_0) \in D\). Supongamos además que \(f\) tiene derivada parcial respecto a \(x\) en todo punto de \(D\)
    y que \(\partial f / \partial x\) es también continua en \(D\). Entonces existe un \(h > 0\) real
    tal que
    \begin{equation}\label{eq:ex_unicidad:teoria:pvi}
        \begin{cases}
            x' = f(t, x) \\
            x(t_0) = x_0
        \end{cases}
    \end{equation}
    tiene una y solo una solución en \(I = [t_0 - h, t_0 + h]\).
\end{theorem}

\begin{proof}
    Como \(R = \{(t, x) : \abs{t - t_0} \leq a, \abs{x - x_0} \leq b\}\) es cerrado y acotado 
    y \(f\) continua en \(D\) podemos definir
    \[
        M = \max_{(t, x) \in R} \abs{f(t, x)}
    \]
    Elegimos entonces \(h = \min\{a, b / M\}\) (veremos ahora por qué) y reducimos \(R\)
    al rectángulo \(R_0 = \{(t, x) : \abs{t - t_0} \leq h, \abs{x - x_0} \leq b\}\).

    Utilizaremos las iteraciones de Picard para obtener una solución del PVI pasando al límite, para ello primero
    comprobamos que todo término de la sucesión \(\{x_n(t)\}\) de iteraciones de Picard es continua y contenida en \(R_0\).

    Sea \(u(t)\) continua en \(I\)
    tal que \(u(t_0) = x_0\) y \((t, u(t)) \in R_0\) para todo \(t \in I\). La función
    \[
        v(t) = x_0 + \int_{t_0}^t f(s, u(s)) \, ds
    \]
    tiene las mismas propiedades.

    En efecto, \((t, u(t)) \in R_0 \subset R\) para \(\abs{t - t_0} \leq h\), por lo que
    \[
        \abs{v(t) - x_0} \leq \abs{\int_{t_0}^t f(s, u(s)) \, ds} \leq M \abs{t - t_0} \leq Mh \leq b
    \]
    por la elección de \(h\).

    Obetenemos así, por inducción, que las iteraciones de Picard \(\{x_n(t)\}\) están bien definidas
    y tienen gráfica contenida en \(R_0\) cuando \(\abs{t - t_0} \leq h\) (por esto reducimos \(R\) a \(R_0\)).

    Observamos ahora que el término \(x_n(t)\) con \(n > 1\) es la suma parcial n-ésima de la serie
    \begin{equation} \label{eq:ex_unicidad:teoria:3}
        x_0(t) + \sum_{k \geq 1} (x_k(t) - x_{k - 1}(t))
    \end{equation}
    con lo que la convergencia de las iteraciones de Picard equivale a la convergencia de la serie~\ref{eq:ex_unicidad:teoria:3}.
    Vamos a probar ahora dos resultados sobre la serie~\ref{eq:ex_unicidad:teoria:3} que corresponden a la existencia y unicidad respectivamente
    de la solución de~\ref{eq:ex_unicidad:teoria:pvi}.
    \begin{enumerate}[(a)]
        \item La serie~\ref{eq:ex_unicidad:teoria:3} converge absolutamente en \(I\) a \(x(t)\).

        Para \(t \in I\) tenemos
        \begin{equation}\label{eq:ex_unicidad:teoria:5}
            \abs{x_1(t) - x_0} = \abs{\int_{t_0}^t f(s, x_0) \, ds} \leq M \abs{t - t_0}
        \end{equation}
        utilizando~\ref{lemma:ex_unicidad:teoria:acotacion} y~\ref{eq:ex_unicidad:teoria:5} obtenemos
        \begin{align*}
            \abs{x_2(t) - x_1(t)} 
                &= \abs{\int_{t_0}^t f(s, x_1(s)) - f(s, x_0) \, ds} 
                \leq K \abs{\int_{t_0}^t \abs{x_1(s) - x_0} \, ds} \\
                &\leq KM \abs{\int_{t_0}^t \abs{s - t_0} \, ds}
                = MK \frac{\abs{t - t_0}^2}{2!}
        \end{align*}
        y procediendo por inducción
        \begin{align*}
            \abs{x_k(t) - x_{k - 1}(t)} \leq M K^{k-1} \frac{\abs{t - t_0}^k}{k!} \leq \frac{M}{K} \frac{{(Kh)}^k}{k!}
        \end{align*}
        O lo que es lo mismo
        \[
            \max_{t \in I} \abs{x_k(t) - x_{k - 1}(t)} \leq \frac{M}{K} \frac{{(Kh)}^k}{k!}
        \]
        y como la serie
        \[
            \sum_{k \geq 1} \frac{M}{K} \frac{{(Kh)}^k}{k!}
        \]
        converge (con suma \(M/K (e^{Kh} - 1)\)) aplicando el criterio M de Weirstrass tenemos que la serie~\ref{eq:ex_unicidad:teoria:3}
        converge uniformemente en \(I\) a una función \(x(t)\) continua. Está función verifica la
        condición inicial, ya que la verifica cada \(x_n(t)\) y además \((t, x(t)) \in R_0\) para cada \(t \in I\).
        Lo comprobamos, dado \(\epsilon > 0\) existe \(N\) tal que
        \[
            \abs{x(t) - x_0} \leq \abs{x(t) - x_N(t)} + \abs{x_N(t) - x_0} < \epsilon + b
        \]
        y puesto que \(\epsilon\) es arbitrario \(\abs{x(t) - x_0} \leq b\) para \(\abs{t - t_0} \leq h\).
        Esto permite probar 
        \begin{equation} \label{eq:ex_unicidad:teoria:6}
            \lim_{k \to \infty} \int_{t_0}^t f(s, x_k(s)) \, ds = \int_{t_0}^t f(s, x(s)) \, ds
        \end{equation}
        En efecto, la convergencia \(x_k(t) \to x(t)\) es uniforme en \(I\) por lo que dado \(\epsilon > 0\) 
        existe \(N\) suficientemente grande tal que
        \[
            \forall s \in I, \, \abs{x_N(s) - x(s)} \leq \epsilon 
        \]
        Entonces, para N suficientemente grande
        \[
            \abs{\int_{t_0}^t f(s, x_N(s)) \, ds - \int_{t_0}^t f(s, x(s)) \, ds} 
                \leq K \abs{\int_{t_0}^t \abs{x_N(s) - x(s)} \, ds}
                \leq K \epsilon \abs{t - t_0}
                \leq K \epsilon h
        \]
        con lo que se da~\ref{eq:ex_unicidad:teoria:6}, lo que implica
        \[
            x(t) = \lim_{k \to \infty} x_{k}(t) = \lim_{k \to \infty} \left(x_0 + \int_{t_0}^t f(s, x_{k-1}(s)) \, ds\right)
            = x_0 + \int_{t_0}^t f(s, x(s)) \, ds
        \]
        y por tanto \(x(t)\) es solución del PVI~\ref{eq:ex_unicidad:teoria:pvi}.

        \item La funcion \(x(t)\) es solución única del PVI en \(I\).
        
        Vemos que si \(y(t)\) es solución del PVI~\ref{eq:ex_unicidad:teoria:pvi} en \(I\) entonces
        necesariamente \(\abs{y(t) - x_0} \leq b\), osea, su gráfica yace en \(R_0\). Para demostrarlo procedemos
        por reducción al absurdo, existe un \(t'\) con \(t_0 < t' < t_0 + h\) tal que \(\abs{y(t') - x_0} > b \geq Mh\).

        Ahora, puesto que \(\abs{y(t) - x_0}\) es continua existe un primer \(t_1\) con \(t_0 < t_1 < t'\) tal que
        \begin{equation}\label{eq:ex_unicidad:teoria:4}
            \abs{y(t_1) - x_0} = b
        \end{equation}
        Como \(y(t)\) es solución del PVI tenemos
        \[
            \abs{y(t_1) - x_0} = \abs{\int_{t_0}^{t_1} f(s, x(s)) \, ds} \leq M \abs{t_1 - t_0} < Mh \leq b
        \]
        que contradice~\ref{eq:ex_unicidad:teoria:4} por la desigualdad estricta.

        Ahora que sabemos que la gráfica de cualquier solución yace en \(R_0\) conssideramos dos soluciones
        \(x\) e \(y\) en \(I\), tenemos
        \[
            \abs{x(t) - y(t)} = \abs{int_{t_0}^t f(s, x(s)) - f(s, y(s)) \, ds} \leq K \abs{\int_{t_0}^t \abs{x(s) - y(s)} \, ds} 
        \]
        donde aplicando el lema de Gronwall~\ref{lemma:ex_unicidad:teoria:gronwall} obtenemos
        que \(\abs{x(t) - y(t)} = 0\).

    \end{enumerate}
\end{proof}

Notamos que las cotas elegidas en el teorema dependen del rectángulo \(R\) elegido al principio, como este
es un resultado local no nos preocupamos de obtener el intervalo más amplio posible, veremos cual es este intervalo
más adelante.

\subsection{Teorema de existencia y unicidad local para funciones Lipschitz}

Notamos que en la demostración anterior la continuidad de \(\partial f / \partial x\) era necesaria solo para
usar la acotacion~\ref{lemma:ex_unicidad:teoria:acotacion}. En este apartado veremos que podemos relajar esta hipótesis
a funciones que cumplan una condición más débil, la condición de Lipschitz.

\begin{definition}[Condición de Lipschitz global]
    Diremos que la funcion \(f(t, x)\) cumple la condición de Lipschitz global (o es globalmente lipschitziana) en un 
    conjunto \(A \subset \mathbb{R}^2\) si existe una constante \(K\) tal que
    \[
        \abs{f(t, x_1) - f(t, x_2)} \leq K \abs{x_1 - x_2} , \quad \forall x_1, x_2 \in A
    \]
    Llamaremos a la constante \(K\) \emph{constante de Lipschitz} de \(f\) en \(A\).
\end{definition}

Es claro que si \(K\) es una constante de Lipschitz para \(f\) también lo es cualquier número mayor que \(K\),
y si \(f\) es Lipschitz respecto a \(x\) es continua respecto a \(x\) fijado \(t\).

Vemos que cumplir la condición de Lipschitz global es un requisito muy fuerte, en la demostración de~\ref{theorem:ex_unicidad:teoria:teorema}
usamos solo la acotación en un rectángulo. Introducimos entonces la siguiente definición.

\begin{definition}[Condición de Lipschitz local]
    Diremos que la funcion \(f(t, x)\) cumple la condición de Lipschitz local (o es localmente lipschitziana) en un 
    conjunto \(D \subset \mathbb{R}^2\) si es Lipschitz en todo rectángulo
    \[
        R = \{(t, x) : \abs{t - t_0} \leq a, \abs{x - x_0} \leq b\} \subset D
    \]
\end{definition}

Es inmediato comprobar que esta es una condición más débil que la continuidad de \(\partial f / \partial x\),
ya que la continuidad de la derivada implica que la funcion es localmente Lipschitziana, quizás con constante 
de Lipschitz distinta en cada rectángulo, pero el recíproco no es cierto. (TODO:\@ contraejemplo)

Con las hipótesis de \(f\) continua y localmente Lipschitz la demostración de~\ref{theorem:ex_unicidad:teoria:teorema}
nos da el resultado cambiando la acotacion. Es posible demostrar (aunque no lo haremos) que con solo suponer
\(f\) continua existe al menos una solucion del PVI~\ref{eq:ex_unicidad:teoria:pvi}, pero
necesitamos hipótesis asicionales sobre \(f\) para garantizar la \emph{unicidad}. Enunciamos el teorema por su interés teórico.

\begin{theorem}[Teorema de Peano de existencia local]
    Sea \(f(t, x)\) continua en un abierto \(D \subset \mathbb{R}^2\) y sea \((t_0, x_0) \in D\).
    Existe un \(h > 0\) tal que el PVI~\ref{eq:ex_unicidad:teoria:pvi} 
    \[
        \begin{cases}
            x' = f(t, x) \\
            x(t_0) = x_0
        \end{cases}
    \]
    tiene al menos una solución en \(I = [t_0 - h, t_0 + h]\).
\end{theorem}

\section{Intervalos máximos de existencia}

En la sección anterior vimos que el teorema de existencia y unicidad local localmente, esto es asegura la existencia
de un \(h > 0\), arbitrariamente pequeño, tal que el PVI~\ref{eq:ex_unicidad:teoria:pvi} tiene solución 
única en \(I = [t_0 - h, t_0 + h]\). Veremos que esta solución local se puede prolongar hasta obtener una 
\emph{solución maximal} que está definida en un intervalo lo más amplio posible.

\begin{definition}[Prolongación de una solución]
    Sea \(u(t)\) una solución de la ecuación diferencial \(x' = f(t, x)\) definida en un intervalo \(I\).
    Diremos que \(v(t)\) es una \emph{prolongación} de \(u(t)\) si \(v\) es solución de la ecuación diferencial
    en un intervalo \(J\) que contiene a \(I\) y \(v(t) = u(t)\) para todo \(t \in I\).
\end{definition}

Consideramos la solución \(x(t)\) definida \(I = [t_0 - h, t_0 + h]\) garantizada por el teorema~\ref{theorem:ex_unicidad:teoria:teorema},
si ponemos ahora \(t_1 = t_0 + h\), \(x_1 = x(t_0 + h)\) el PVI
\[
    \begin{cases}
        x' = f(t, x) \\
        x(t_1) = x_1
    \end{cases}
\]
tiene por el teorema~\ref{theorem:ex_unicidad:teoria:teorema} 
una solución única \(x_1(t)\) definida en un intervalo \(I_1 = [t_1 - h_1, t_1 + h_1]\), para cierto \(h_1 > 0\).
Por el resultado de unicidad ha de ser \(x(t) = x_1(t)\) para \(t \in [a, t_0 + h]\) donde \(a = \max\{t_0 - h, t_1 - h_1\}\).
Así la función \(v(t)\) definida como
\[
    v(t) =
    \begin{cases}
        x(t), \quad t \in [t_0 - h, t_0 + h] \\
        x_1(t), \quad t \in [t_0 - h, t_0 + h + h_1]
    \end{cases}
\]
es una prolongación (a la derecha) de \(x(t)\) (análogamente se obtendría una prolongación a la izquierda).
Repitiendo este proceso obtendríamos una solución en intervalos cada vez más grandes ¿hasta cuando podemos continuar?

\end{document}