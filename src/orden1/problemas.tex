\documentclass[../ecuaciones_diferenciales.tex]{subfiles}

\begin{document}

\begin{problem}
Encontrar la solución general de las siguientes ecuaciones diferenciales,
indicando su intervalo de definición.

\begin{multicols}{2}
	\begin{enumerate}[a)]
		\item \(\displaystyle x' + x \cos t = 0\)
		\item \(\displaystyle x' + x\sqrt{t} \sin t = 0\)
		\item \(\displaystyle x' + \frac{2t}{1 + t^2}x = \frac{1}{1 + t^2}\)
		\item \(\displaystyle x' + x = t e^t\)
	\end{enumerate}
\end{multicols}
\end{problem}

\begin{solution}
	\begin{enumerate}[a), wide, labelwidth=0pt, labelindent=0pt]
		\item Se trata de una ecuación homogénea, puesto que
		      \(\cos t \in C(\R)\), el intervalo de definición de la ecuación es
		      \(\R\). Separando variables:
		      \begin{align*}
			      \frac{x'}{x} = -\cos t & \iff
			      \log\abs{x(t)} = \int_0^t -\cos s \dif s + c
			      = -(\sin t - \sin 0) + c                            \\
			                             & \iff x(t) = k e^{-\sin t}.
		      \end{align*}

		\item El intervalo de definición es \(\R^+ \cup \set{0}\), utilizando la
		      fórmula obtenida en~\ref{sec:hom}:
		      \[x(t) = k e^{\int^t \sqrt{s} \sin s \dif s},\]
		      la integral no posee una primitiva elemental, por lo que no podemos
		      simplificar más la solución.

		\item Tanto \(a(t)\) como \(b(t)\) son continuas en \(\R\). Utilizamos
		      el método de variación de constantes, resolvemos primero la ecuación
		      homogénea asociada:
		      \[x_n(t) = k e^{\int_0^t \frac{-2s}{1 + s^2} \dif s},\]
		      notamos que la derivada de \(-\log(1 + s^2)\) coincide con el
		      interior de nuestra integral, por tanto:
		      \[x_n(t) = k e^{-\log(1 + t^2)} = \frac{k}{1 + t^2}.\]

		      Ahora, sea \(x_p = \frac{k(t)}{1 + t^2}\) solución de la ecuación
		      diferencial, entonces
		      \[x_p'(t) = k'(t)\frac{1}{1 + t^2} + k(t)\frac{-2t}{(1 + t^2)^2}\]
		      sustituyendo en la ecuación obtenemos que
		      \[x_p'(t) = \frac{-2t}{1 + t^2} x_p(t) + \frac{1}{1 + t^2}
			      = k(t)\frac{-2t}{(1 + t^2)^2}  + \frac{1}{1 + t^2}.\]

		      Igualando hallamos que \(k'(t) = 1\) y, por ende, \(k(t) = t\).
		      La solución general es
		      \[x(t) = x_n(t) + x_p(t) = \frac{k + t}{1 + t^2}.\]

		\item Tanto \(a = 1\) como \(b\) son continuas en todo \(\R\) por lo que
		      este es nuestro intervalo de definición. Resolvemos la ecuación
		      homogénea asociada, obteniendo \(x_n(t) = k e^{-t}\). Sea
		      \(x_p(t) = k(t) e^{-t}\) solución de la ecuación, entonces:
		      \[\begin{cases}
				      x_p'(t) = k'(t) e^{-t} - k(t) e^{-t} \\
				      x_p'(t) = t e^t - x_p(t) = t e^t - k(t) e^{-t}
			      \end{cases}
			      \leadsto
			      k'(t) = t e^{2t}.
		      \]
		      Integrando por partes:
		      \[k(t) = \int t e^{2t} \dif t
			      = \frac{t e^{2t}}{2} - \int \frac{e^{2t}}{2} \dif t
			      = \frac{t e^{2t}}{2} - \frac{e^{2t}}{4}.\]
		      La solución general de la ecuación es pues
		      \[x(t) = x_n(t) + x_p(t) = (k + k(t))e^{-t}
			      = k e^{-t} + \frac{t e^{t}}{2} - \frac{e^{t}}{4}.\]
	\end{enumerate}
\end{solution}

\begin{problem}
Hallar las soluciones de los siguientes problemas de valor inicial,
indicando su intervalo de definición.
\begin{multicols}{2}
	\begin{enumerate}[a)]
		\item \(\displaystyle
		      \begin{cases}
			      x' + 5x = t^2 \\
			      x(0) = 3
		      \end{cases}\)

		\item \(\displaystyle
		      \begin{cases}
			      x' = (\tan t)x + \cos t \\
			      x(0) = 1
		      \end{cases}\)

		\item \(\displaystyle
		      \begin{cases}
			      x' + 2tx = t^3 \\
			      x(0) = 1
		      \end{cases}\)

		\item \(\displaystyle
		      \begin{cases}
			      tx' + \frac{t}{\sqrt{1 + t^3}}x = 1 \\
			      x(1) = 2
		      \end{cases}\)
	\end{enumerate}
\end{multicols}
\end{problem}

\begin{solution}
	\begin{enumerate}[a), wide, labelwidth=0pt, labelindent=0pt]
		\item Resolvemos la ecuación homogénea asociada, \(x_n(t) = k e^{-5t}\),
		      y, mediante variación de constantes, obtenemos, integrando por
		      partes dos veces,
		      \[k(t) = \parens{\frac{t^2}{5} - \frac{2t}{25}
				      + \frac{2}{125}} e^{-5t},\]
		      así una solución general es
		      \[x(t) = (k + k(t)) e^{-5t}
			      = k e^{-5t} + \frac{t^2}{5} - \frac{2t}{25} + \frac{2}{125}.\]

		      Evaluamos en la condición inicial
		      \[3 = x(0) = k + \frac{2}{125} \leadsto k = \frac{373}{125},\]
		      obtenemos así la solución del problema del valor inicial
		      \[x(t) =  \frac{373}{125} e^{-5t} + \frac{t^2}{5} - \frac{2t}{25}
			      + \frac{2}{125}.\]

		\item TODO.

		\item Resolvemos la ecuación homogénea asociada separando variables,
		      \(x = k e^{-t^2}\), por variación de constantes,
		      \(k'(t) = t^3 e^{t^2}\) y, por partes,
		      \(k(t) = 1/2 (e^{t^2}t^2 - e^{t^2})\). La solución general es pues
		      \[x(t) = k e^{-t^2} + \frac{1}{2}(t^2 - 1).\]

		      Evaluando en la condición inicial, \(1 = x(0) = k - 1/2\), por lo
		      que \(k = 1/2\).

		\item El intervalo de definición es \((0, \infty)\), ya que es el
		      intervalo más grande en el cual las funciones \(a\) y \(b\) son
		      continuas. Separando y puesto que el integrando no posee
		      primitiva elemental:
		      \[x = k e^{-\int^t \frac{1}{\sqrt{1 + s^3}} \dif s}.\]

		      Por variación de constantes,
		      \[k(t) = \int^t \frac{1}{s}
			      e^{\int^s \frac{1}{\sqrt{1 + u^3}} \dif u} \dif s,\]
		      con lo cual
		      \[x(t) = (k + k(t))  e^{-\int^t \frac{1}{\sqrt{1 + s^3}} \dif s}.\]

		      Obtenemos la solución del problema del valor inicial evaluando,
		      \(x(1) = k = 2\).
	\end{enumerate}
\end{solution}

\begin{problem}
El nivel de carbón vegetal encontrado en las grutas de Lascaux, en Francia,
dio una medida de 0,91 desintegraciones por minuto y gramo (la madera viva
da 6,68 desintegraciones). Estimad la edad probable de las pinturas que se
hallan en dichas cuevas.
\end{problem}

\begin{solution}
	Elegimos \(t = 0\) el momento en el que se corta el árbol, y denotamos por
	\(x(t)\) la cantidad de carbono-14 a tiempo \(t\), por lo que
	\(x'(0) = -6,68 q\), donde \(q\) es la cantidad de carbono-14 en 1 gramo de
	madera.  Utilizando la fórmula vista en clase,
	\[x(t) = x(0) e^{-\lambda t},\]
	donde \(\lambda = \frac{\log 2}{5730}\), deducimos que
	\[x'(t) = -\lambda x(0) e^{-\lambda t}.\]

	Según los datos del enunciado, \(-6,68 q = x'(0) = -\lambda x(0)\). Si
	llamamos \(t_\alpha\) al momento en el que estamos, tenemos
	\(-0,91 q = x'(t_\alpha) = - \lambda x(0) e^{-\lambda t_\alpha}\), y
	dividiendo queda
	\[\frac{0,91}{6,68} = \frac{x'(t_\alpha)}{x'(0)} = e^{-\lambda t_\alpha},\]
	de donde se deduce que \(t_\alpha = 16479\), es decir, las pinturas tienen una
	antigüedad de 16479 años.
\end{solution}

\end{document}
