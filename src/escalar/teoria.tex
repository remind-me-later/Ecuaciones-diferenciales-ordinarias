\documentclass[../ecuaciones_diferenciales.tex]{subfiles}

\begin{document}

\begin{definition}
	\label{def:sisescordsup}
	Una ecuación diferencial escalar de orden superior es una ecuación de la
	forma:
	\[x^{(n)} = a_n(t) x^{(n - 1)} + \cdots + a_1(t)x + a_0(t).\]
\end{definition}

Definiendo las variables \(x_n = x^{(n - 1)}\), o lo que es lo mismo,
\(x_n = x'_{n - 1}\), el sistema~\ref{def:sisescordsup} es equivalente a:
\[\eqsys{
	x_2 &= x'_1 \\
	x_3 &= x'_2 \\
	&\vdots \\
	x_n &= x'_{n - 1} \\
	x'_n &= a_n(t) x_n + \cdots + a_1(t) x_1 + a_0(t)
	}\]
donde la última ecuación es de primer orden, podemos también expresar este
sistema matricialmente como \(\vec{x}' = A(t)\vec{x} + \vec{b}(t)\) con:
\[A(t) = \mat{
		0 & 1 & & \\
		& \ddots & \ddots & \\
		& & 0 & 1 \\
		a_1(t) & a_2(t) & \cdots & a_n(t)
	}
	\quad \text{y} \quad
	\vec{b}(t) = \mat{0 \\ \vdots \\ 0 \\ a_0(t)}.
\]

En el caso del sistema el problema del valor inicial consiste en fijar 
\(x_1(t_0), x_2(t_0), \dots, x_n(t_0)\), lo que equivale aquí a fijar 
\(x(t_0), x'(t_0), \dots, x^{(n-1)}(t_0)\). 
Aplicando lo visto anteriormente el sistema tiene
solución general dada por \(\vec{x}_p(t) + \vec{x}_h(t)\) donde, 
como de costumbre, \(\vec{x}_p(t)\) es una solución particular y 
\(\vec{x}_h(t)\) es la solución general de la ecuación homogénea asociada, 
la cual es un espacio vectorial de dimensión \(n\).

Si \(\vec{\phi}_1, \dots, \vec{\phi}_n\) son soluciones linealmente 
independientes de~\ref{def:sisescordsup} la matriz fundamental asociada al 
sistema equivalente es
\[\Phi(t) = \mat{
		\phi_1 & \dots & \phi_n \\
		\phi'_1 & \dots & \phi'_n \\
		\vdots & & \vdots \\
		\phi^{(n - 1)}_1 & \dots & \phi^{(n - 1)}_n}.\]

\begin{definition}[Wronskiano]
	Llamaremos Wronskiano de \(\phi_1, \dots, \phi_n\) al determinante
	\[W(\phi_1, \dots, \phi_n)(t) = \det\mat{
			\phi_1(t) & \dots & \phi_n(t) \\
			\phi'_1(t) & \dots & \phi'_n(t) \\
			\vdots & & \vdots \\
			\phi^{(n - 1)}_1(t) & \dots & \phi^{(n - 1)}_n(t) \\
		}, \quad t \in (\alpha, \omega).\]
\end{definition}

\section{Matriz fundamental}

Hemos dejado pendiente cómo obtener una matriz fundamental \(\Phi(t)\) asociada
al sistema homogéneo \(\vec{x}' = A(t)\vec{x}\), por ahora nos restringiremos 
al caso de coeficientes constantes, es decir, \(\vec{x}' = A\vec{x}\) 
donde \(A\) es una matriz constante, no depende de \(t\) y nos centraremos en 
particular en funciones de dominio \(\R\).

En el caso escalar, \(x' = ax\), sabemos que la solución es \(x = ke^{at}\),
donde podemos interpretar \(e^{at}\) como una 
\textquote{matriz fundamental \(1 \times 1\)}. 
En el caso matricial veremos que una matriz fundamental es
\(\Phi(t) = e^{At}\), aunque primero tendremos que definir la exponencial de una
matriz y obtener métodos para computarla.

\begin{lemma}
	\label{lem:expphi}
	Sea \(\Phi(t)\) la matriz fundamental asociada a las soluciones de
	\[\eqsys{
		\vec{x}' = A\vec{x} \\
		\vec{x}(0) = \vec{e}_j}\]
	o, equivalentemente, la única solución del problema de valor inicial
	\[\eqsys{
			X' = AX \\
			X(0) = \mathit{Id}}\]

	Se dan las siguientes igualdades:
	\begin{multicols}{3}
		\begin{enumerate}[i)]
			\item \(\displaystyle \Phi(0) = \mathit{Id}\)

			\item \(\displaystyle \Phi(t + s) = \Phi(t) \Phi(s)\)

			\item \(\displaystyle \Phi(-t) = \Phi^{-1}(t)\)
		\end{enumerate}
	\end{multicols}
\end{lemma}

\begin{proof}
	\begin{enumerate}[i), wide, labelwidth=0pt, labelindent=0pt]
		\item La primera igualdad se sigue directamente de la definición.

		\item Fijamos \(s\) y consideramos el problema de valor inicial
			\[\eqsys{\vec{x}' = A\vec{x} \\ \vec{x}(0) = \Phi(s) }\]
		      Vamos a ver que tanto \(\Phi(t + s)\) como \(\Phi(t)\Phi(s)\) (como
		      funciones en \(t\)) son soluciones de ese problema. Consideramos
		      primero \(X(t) := \Phi(t + s)\), tenemos que \(X(0) = \Phi(s)\) y
		      \[X'(t) = \Phi'(t + s) = A \Phi(t + s) = A X(t),\]
		      similarmente para \(Y(t) := \Phi(t)\Phi(s)\), aplicando la primera
		      igualdad \(Y(0) = \Phi(0)\Phi(s) = \Phi(s)\) e
		      \[Y'(t) = \Phi'(t)\Phi(s) = A \Phi(t)\Phi(s) = A Y(t).\]

		\item Aplicando la primera y segunda igualdad en los respectivos
		      miembros:
		      \[\mathit{Id} = \Phi(t + (-t)) = \Phi(t)\Phi(-t).\]
	\end{enumerate}
\end{proof}

Para la matriz fundamental \(\Phi\) así definida, 
\(\vec{x}(t) = \Phi(t)\vec{x}^0\) es la solución del sistema
\[\eqsys{\vec{x}' = A\vec{x} \\ \vec{x}(0) = \vec{x}^0}\]

\section{Exponencial de una matriz}

Notamos que la matriz que acabamos de definir \(\Phi\) tiene muchas de las
propiedades de la función exponencial, de hecho más adelante demostraremos que
\(\Phi(t) = \sum_{k=0}^\infty (tA)^k/k!\), por ahora nos contentamos con la
siguiente definición:

\begin{definition}[Exponencial de una matriz]
	Dada una matriz \(B\) se define \(\exp(B)\), o con la notación habitual
	\(e^B\), como
	\[e^B = \sum_{k = 0}^\infty \frac{B^k}{k!}.\]
\end{definition}

Presentamos ahora algunas propiedades típicas de la función exponencial que
también cumple esta versión matricial, la mayoría son corolario de lo visto
arriba.

\begin{proposition}
	Se cumplen las siguientes igualdades.
	\begin{multicols}{2}
		\begin{enumerate}[i)]
			\item \(\displaystyle e^0 = \mathit{Id}\)

			\item \(\displaystyle e^{(t + s)A} = e^{tA} e^{sA}\)

			\item \(\displaystyle \parens{e^{tA}}^{-1} = e^{-tA}\)

			\item \(\displaystyle \frac{d}{\dif t} e^{tA} = A e^{tA}\)
		\end{enumerate}
	\end{multicols}
\end{proposition}

\begin{remark}
	No es cierto en general que \(e^{A + B} = e^A e^B\), solo se cumple si las
	matrices \(A\) y \(B\) conmutan.
\end{remark}

\begin{theorem}
	La matriz \(\Phi(t)\) definida en~\ref{lem:expphi} es igual a \(e^{At}\),
	dicho de otro modo:
	\[\Phi(t) = \sum_{k = 0}^\infty \frac{t^k A^k}{k!} =
		\sum_{k = 0}^\infty \frac{(tA)^k}{k!}.\]
\end{theorem}

\begin{proof}
	Haremos uso de las iteraciones de Picard, para ello definimos el operador
	\begin{align*}
		T : C((\alpha, \omega), \mathcal{M}_{n \times n}) 
		& \to C((\alpha, \omega), \mathcal{M}_{n \times n}) \\
		\Psi & \mapsto \mathit{Id} + \int_0^t A \Psi(s) \dif s.
	\end{align*}

	Comenzamos la iteración con \(X_0(t) = \mathit{Id}\) y proseguimos:
	\begin{align*}
		X_1 & = T(X_0) = \mathit{Id} + \int_0^t A \mathit{Id} \dif s =
		\mathit{Id} + At                                                      \\
		X_2 & = T(X_1) = \mathit{Id} + \int_0^t A (\mathit{Id} + As) \dif s =
		\mathit{Id} + At + \frac{A^2 t^2}{2}                                  \\
		    & \vdots                                                          \\
		X_n & = T(X_{n - 1}) =
		\mathit{Id} + At + \frac{A^2 t^2}{2} + \cdots + \frac{A^n t^n}{n!}
		= \sum_{k = 0}^n \frac{A^k t^k}{k!},
	\end{align*}
	tomando el límite \(\lim_{n \to \infty} X_n\) obtenemos el resultado.
\end{proof}

\section{Métodos de cómputo}

Con los resultados vistos arriba lo único que necesitamos para ser capaces de
resolver una ecuación escalar de orden superior son métodos para computar 
\(e^{A t}\) para \(A \in \mathcal{M}_{n \times n}(\R)\) arbitraria. 
A continuación presentamos métodos para distintos tipos de matrices, empezando
por el más concreto para terminar con uno válido para cualquier matriz cuadrada
real, cada uno de ellos se basa en el anterior, por lo que es recomendable
leer todos.

\subsection{Matrices diagonales}

Comenzamos con el caso más simple, una matriz diagonal, una vez visto este el
caso diagonalizable se sigue casi directamente.

\begin{proposition}
	Sea \(A\) una matriz diagonal, podemos expresar \(e^{At}\) como:
	\[e^{At} = \exp\mat{\lambda_1t & & \\ & \ddots & \\ & & \lambda_nt}
		= \mat{e^{\lambda_1 t} & & \\ & \ddots & \\ & & e^{\lambda_n t}}.\]
\end{proposition}

Presentamos dos demostraciones distintas, la primera usa resultados anteriores,
la segunda es directa.

\begin{proof}
	Sea \(A\) una matriz diagonal, entonces:
	\[A = \mat{
			\lambda_1 & & \\
			& \ddots & \\
			& & \lambda_n}
		\quad \text{y} \quad
		\vec{x}' = A\vec{x} \iff
		\eqsys{
			x'_1 &= \lambda_1 x_1 \\
			&\vdots \\
			x'_n &= \lambda_n x_n \\
		}
	\]
	La solución general de este sistema viene dada por
	\(x_1(t) = k_1 e^{\lambda_1 t}, \dots, x_n(t) = k_n e^{\lambda_n t}\) o, 
	lo que es lo mismo,
	\[\vec{x} = \mat{e^{\lambda_1 t} & & \\ & \ddots & \\ & & e^{\lambda_n t}}
		\mat{k_1 \\ \vdots \\ k_n}.\]
	Puesto que es solución del problema de valor inicial
	\(\set{X' = AX,\ X(0) = \mathit{Id}}\) hemos comprobado a posteriori, por
	la unicidad de la solución, que la matriz de la izquierda es \(e^{At}\). 
\end{proof}

Para convencernos lo comprobamos también directamente.

\begin{proof}
	Tenemos que
	\[e^{At} = \exp\smat{\lambda_1t & & \\ & \ddots & \\ & & \lambda_nt}
		= \sum_{k = 0}^\infty
		\frac{1}{k!}\smat{\lambda_1 & & \\ & \ddots & \\ & & \lambda_n}^k t^k
		= \sum_{k = 0}^\infty
		\frac{1}{k!} \smat{\lambda_1 t & & \\ & \ddots & \\ & & \lambda_n t}^k,
	\]
	ahora, explotando la diagonalidad:
	\[\sum_{k = 0}^\infty \frac{1}{k!} 
			\smat{\lambda_1 t & & \\ & \ddots & \\ & & \lambda_n t}^k
		= \sum_{k = 0}^\infty \frac{1}{k!} 
			\smat{(\lambda_1 t)^k & & \\ & \ddots & \\ & & (\lambda_n t)^k}
		= \mat{\sum_{k = 0}^\infty \frac{(\lambda_1 t)^k}{k!} & & \\
			& \ddots &
			\\ & & \sum_{k = 0}^\infty \frac{(\lambda_n t)^k}{k!}}.\]
\end{proof}

\subsection{Matrices diagonalizables con autovalores reales}

Sea \(A\) una matriz diagonalizable con autovalores reales
\(\lambda_1, \dots, \lambda_n\). 
Por álgebra lineal sabemos que si \(\{\vec{v}_1, \vec{v}_2, \dots, \vec{v}_n\}\)
es una base de autovectores asociados, entonces
\[P^{-1} A P = D = \mat{\lambda_1 && \\ & \ddots & \\ && \lambda_n}
	\quad \text{siendo} \quad
	P = \mat{\vec{v}_1 & \vec{v}_2 & \dots & \vec{v}_n}.\]

Gracias a esta descomposición podremos demostrar el siguiente teorema.

\begin{proposition}
	Sea \(A \in \mathcal{M}_{n \times n}(\R)\) diagonalizable con autovalores 
	reales. La solución general del sistema \(\vec{x}' = A\vec{x}\) es
	\[\vec{x}(t) = \mat{k_1 e^{\lambda_1 t} \\ \vdots \\ k_n e^{\lambda_n t}},
	\quad \mat{k_1 \\ \vdots \\ k_n} \in \R^n.\]
\end{proposition}

\begin{proof}
	Puesto que \(A\) es diagonalizable la ecuación \(\vec{x}' = A\vec{x}\) 
	se puede escribir como
	\(\vec{x}' = P D P^{-1} \vec{x}\). Con el cambio de variable 
	\(\vec{y} = P^{-1}\vec{x}\), nos queda
	\[\vec{x}' = A\vec{x} \iff 
		\vec{x}' = PDP^{-1}\vec{x} \iff 
		P^{-1}\vec{x}' = DP^{-1}\vec{x} \iff \vec{y}' = D\vec{y},\]
	donde es fundamental que \(P\) (y por tanto \(P^{-1}\)) sea una matriz constante
	para poder afirmar que \(\vec{y}' = (P^{-1}\vec{x})' = P^{-1}\vec{x}'\). 
	Con este cambio de variable podemos expresar la ecuación como
	\[\vec{y}' = \mat{\lambda_1 && \\ & \ddots & \\ && \lambda_n} \vec{y} \iff
		\vec{y} = e^{Dt}\vec{k} 
		= \mat{k_1 e^{\lambda_1 t} \\ \vdots \\ k_n e^{\lambda_n t}}.\]
\end{proof}

Tras estos preámbulos podemos obtener
una expresión explícita del espacio de soluciones
\begin{align*}
	\vec{x} &= P\vec{y} = \mat{\vec{v}_1 & \dots & \vec{v}_n} 
		\mat{k_1 e^{\lambda_1 t} \\ \vdots \\ k_n e^{\lambda_n t}} =
	k_1 e^{\lambda_1 t} \vec{v}_1 + \cdots + k_n e^{\lambda_n t}\vec{v}_n \\
	  &= P \mat{e^{\lambda_1 t} & & \\ & \ddots & \\ & & e^{\lambda_n t}}
      \mat{k_1 \\ \vdots \\ k_n} = P e^{D t} P^{-1} \mat{c_1 \\ \vdots \\ c_n} =
      e^{A t} \mat{c_1 \\ \vdots \\ c_n}
\end{align*}

\begin{remark}
	El espacio de soluciones es el espacio vectorial generado por
	\(\set{e^{\lambda_1 t} \vec{v}_1, \dots, e^{\lambda_n t} \vec{v}_n}\).
\end{remark}

\subsection{Matrices diagonalizables con autovalores complejos}

Consideramos ahora el caso diagonalizable general, en el que \(A\) puede tener
autovalores complejos. En estas condiciones, ya sabemos resolver 
\(\vec{x}' = A\vec{x}\) en \(\Complex\), es decir, si permitimos como solución 
\(\vec{z}(t) = \vec{x}(t) + \iu\vec{y}(t)\) porque todo lo
que hemos visto funciona igual para números complejos, pero puesto que nuestra 
matriz original solo tenía coeficientes reales no es descabellado asumir que lo 
que se busca son soluciones reales.

La siguiente proposición nos muestra una curiosa
propiedad de los autovalores complejos que explotaremos para obtener una matriz
semejante a la original cuya exponencial sea fácilmente computable.

\begin{proposition}
	Si \(A\) tiene coeficientes reales y \(\lambda\) es un autovalor complejo 
	de \(A\), entonces su conjugado \(\conjugate{\lambda}\) también lo es.
\end{proposition}

\begin{proof}
	El polinomio característico de \(A\), \(p(\lambda) = \det(A - \lambda I)\)
	tiene coeficientes reales, es decir, \(p(\lambda) = \sum_{i=1}^n
	a_i\lambda^i\) con \(a_i \in \R\).

	Si \(\lambda_0 \in \Complex\) es raíz del polinomio, es decir, \(\sum_{i=1}^n
	a_i\lambda_0^i = 0\), entonces

	\[0 = \conjugate{0} = \conjugate{\parens{\sum_{i=1}^n a_i\lambda_0^i}} 
		= \sum_{i=1}^n \conjugate{a_i}(\conjugate{\lambda_0})^i 
		= \sum_{i=1}^n a_i(\conjugate{\lambda_0})^i 
		= p(\conjugate{\lambda_0}).\]
\end{proof}

Por la proposición anterior los autovalores no reales siempre vienen por pares,
por lo que nos restringimos de momento a subespacios de dimensión 2.

\begin{proposition}
	Si \(\lambda = a + \iu b\) es un autovalor complejo de \(A\) y 
	\(\vec{w} = \vec{u} + \iu\vec{v}\)
	un autovector asociado, entonces \(\conjugate{\lambda} = a - \iu b\) 
	es también autovalor y \(\conjugate{\vec{w}} = \vec{u} - \iu \vec{v}\) es un 
	autovector asociado.
\end{proposition}

\begin{proof}
	Como \(\lambda\) es un autovalor con autovector \(\vec{w}\), se cumple
	\(A\vec{w} = \lambda \vec{w}\), por lo que
	\[\conjugate{(A\vec{w})} = \conjugate{(\lambda \vec{w})} 
		= \conjugate{\lambda} \conjugate{\vec{w}}
		\quad \text{y} \quad
	\conjugate{(A\vec{w})} = \conjugate{A} \conjugate{\vec{w}} 
		= A \conjugate{\vec{w}},\]
	luego \(A \conjugate{\vec{w}} = \conjugate{\lambda} \conjugate{\vec{w}}\).
\end{proof}

Vamos a ver cómo aprovechar esto para transformar una solución con números 
complejos en otra equivalente que involucre solamente números reales. 

\begin{proposition}
	Sea \(A \in \mathcal{M}_{2 \times 2}(\R)\) diagonalizable con un autovalor 
	complejo \(\lambda = a + \iu b\) y un autovector asociado 
	\(\vec{w} = \vec{u} + \iu\vec{v}\) entonces la solución general del sistema
	\(\vec{x}' = A\vec{x}\) es
	\[\vec{x}(t) = \mat{\vec{u} & \vec{v}} 
		\mat{e^{at}\cos bt & e^{at}\sin bt \\ -e^{at} \sin bt & e^{at}
		\cos bt} \mat{c_1 \\ c_2}, \quad \mat{c_1 \\ c_2} \in \R^2.\]
\end{proposition}

\begin{proof}
	Puesto que los autovalores complejos vienen en pares, \(A\) tiene como
	autovalores \(\lambda = a + \iu b\) y 
	\(\conjugate{\lambda} = a - \iu b \in \Complex\), 
	y autovectores asociados 
	\(\vec{w} = \vec{u} + \iu\vec{v}\) y 
	\(\conjugate{\vec{w}} = \vec{u} - \iu\vec{v}\). Se tiene que
	\[A\vec{u} + \iu A\vec{v} 
		= A\vec{w} = \lambda \vec{w} 
		= (a + \iu b)(\vec{u} + \iu \vec{v}) 
		= (a\vec{u} - b\vec{v}) + \iu(a\vec{v} + b\vec{u}),\]
	de donde, igualando partes real e imaginaria,
	\[\eqsys{
		A\vec{u} = a\vec{u} - b\vec{v} \\
		A\vec{v} = b\vec{u} + a\vec{v}}\]
	Ya hemos visto que \(\vec{u}\) y \(\vec{v}\) son linealmente independientes,
	por lo que la matriz \(P = \mat{\vec{u} & \vec{v}}\) 
	es una matriz de paso. Entonces,
	\[A \mat{\vec{u} & \vec{v}} 
		= \mat{A\vec{u} & A\vec{v}} 
		= \mat{a\vec{u} - b\vec{v} & b\vec{u} + a\vec{v}} 
		= \mat{\vec{u} & \vec{v}} \mat{a & b \\ -b & a},\]
	llamando \(B\) a la matriz de la derecha:
	\[B := \mat{a & b \\ -b & a}\]
	sin más que multiplicar por \(P^{-1}\) a la izquierda del primer y último
	miembro tenemos que las matrices \(A\) y \(B\) son semejantes, 
	\(P^{-1}AP = B\). 
	Sabemos que las soluciones de este sistema vienen dadas por 
	\(\vec{x}(t) = e^{At} \vec{k}\), como \(A = PBP^{-1}\) podemos escribir 
	\[\vec{x}(t) = e^{At} \mat{k_1 \\ k_2} 
		= P e^{Bt} P^{-1} \mat{k_1 \\ k_2} 
		= P e^{Bt} \mat{c_1 \\ c_2}.\]
	Así para conocer las soluciones reales del sistema solo necesitamos calcular
	\begin{align*}
		e^{Bt} &= \exp\mat{at & bt \\ -bt & at} 
			= \exp\parens{a\mat{1 & 0 \\ 0 & 1}t + b\mat{0 & 1 \\ -1 & 0}t} \\
			&= \mat{e^{at}  & 0 \\ 0 & e^{at}} 
			\exp\left(b\mat{0 & 1 \\ -1 & 0}t\right) 
			= e^{at}\exp\left(b\mat{0 & 1 \\ -1 & 0}t\right).
	\end{align*}
	
	Hemos simplificado nuestro problema al de resolver el caso 
	\(\vec{y}' = M\vec{y}\), donde
	\[M := \mat{0 & 1 \\ -1 & 0},\]
	es decir, calcular \(e^{Mt}\).
	Para ello, diagonalizamos \(M\); su polinomio característico es 
	\(p(\lambda) = \lambda^2 + 1\), luego sus
	autovalores son \(\pm \iu\). Los espacios invariantes son, respectivamente,
	\[\ker(M - \iu I) = L\left[\mat{1 \\ \iu}\right] 
		\quad \text{y} \quad 
		\ker(M + \iu I) = L\left[\mat{\iu \\ 1}\right],\]
	con lo que una posible matriz de paso es 
	\[P := \mat{1 & \iu \\ \iu & 1} \quad \text{con} \quad 
		P^{-1} = \frac{1}{2} \mat{1 & -\iu \\ -\iu & 1}.\]
	Se tiene entonces:
	\begin{align*}
		\exp \mat{0 & t \\ -t & 0} &= P^{-1} \exp\mat{\iu & 0 \\ 0 & -\iu} P =
		\frac{1}{2} \mat{1 & \iu \\ \iu & 1}
		\mat{e^{\iu t} & 0 \\ 0 & e^{-\iu t}} \mat {1 & -\iu \\ -\iu & 1} \\
		&= \frac{1}{2} 
		\mat{e^{\iu t} + e^{-\iu t} & -\iu e^{\iu t} + \iu e^{-\iu t} \\
		\iu e^{\iu t} - \iu e^{-\iu t} & e^{\iu t} + e^{-\iu t}} 
		= \mat{\cos t & \sin t \\ -\sin t & \cos t}.
	\end{align*}

	Juntando todo tenemos que las soluciones reales de \(\vec{x}' = A\vec{x}\) 
	son
	\[\vec{x}(t) = P 
		\mat{e^{at}\cos bt & e^{at}\sin bt \\
			-e^{at} \sin bt & e^{at} \cos bt} P^{-1} 
		\mat{k_1 \\ k_2} 
		= P \mat{e^{at}\cos bt & e^{at}\sin bt \\
			-e^{at} \sin bt & e^{at} \cos bt} \mat {c_1 \\ c_2}.\]
\end{proof}

Damos una demostración alternativa, ligeramente más directa, pero menos 
elemental.

\begin{proof}
	Sabemos, por lo visto anteriormente, que 
	\(\vec{z}(t) = e^{\lambda t} \vec{w}\) y
	\(\conjugate{\vec{z}}(t) = e^{\conjugate{\lambda} t} \conjugate{\vec{w}}\)
	son soluciones de \(\vec{x}' = A\vec{x}\) 
	linealmente independientes. Sin más que desarrollar los productos, se
	llega a:
	\begin{align*}
		\vec{z}(t) &= e^{(a + \iu b)t} (\vec{u} + \iu\vec{v}) 
		= e^{at} ((\vec{u}\cos bt - \vec{v}\sin bt) 
		+ \iu(\vec{v}\cos bt +\vec{u}\sin bt)) \\
		\conjugate{\vec{z}}(t) &= e^{(a - \iu b)t} (\vec{u} - \iu\vec{v}) 
		= e^{at} ((\vec{u}\cos bt - \vec{v}\sin bt) 
		- \iu(\vec{v}\cos bt + \vec{u}\sin bt)).
	\end{align*}
	Ahora bien, como:
	\[\mat{\Re \vec{z}(t) \\ \Im \vec{z}(t)} 
		= \frac{1}{2} \mat{\vec{z}(t) + \conjugate{\vec{z}}(t) \\
			(\vec{z}(t) - \conjugate{\vec{z}}(t))/\iu} 
		= \frac{1}{2}\mat{1 & 1 \\ -\iu & \iu} 
		\mat{\vec{z}(t) \\ \conjugate{\vec{z}}(t)} 
		\quad \text{y} \quad 
		\det\mat{1 & 1 \\ -\iu & \iu} \neq 0,\]
	se cumple que \(\Re \vec{z}(t)\) e \(\Im \vec{z}(t)\) siguen siendo 
	soluciones y linealmente independientes, que además generan el mismo 
	espacio de soluciones. Tenemos entonces las soluciones reales:
	\[\eqsys{
		\vec{x}(t) = \Re\vec{z} = e^{at}(\vec{u}\cos bt - \vec{v}\sin bt) \\
		\vec{y}(t) = \Im\vec{z} = e^{at}(\vec{v}\cos bt + \vec{u}\sin bt)}\]
\end{proof}

Para resolver el caso diagonalizable general, es decir, matrices de dimensión
arbitraria con autovalores reales y complejos, vamos a combinar todo lo que 
hemos visto. 

Sea \(A\) una matriz con autovalores reales \(\lambda_1, \dots, \lambda_r\)
y autovalores complejos \(a_1 \pm \iu b_1, \dots, a_s \pm \iu b_s\) con 
\(b_j > 0\), donde en ambos casos se admiten repeticiones. 
Definimos las matrices \(B_1, \dots, B_s\) y la matriz \(P\) como:
\[B_j := \mat{a_j & b_j \\ -b_j & a_j} 
	\quad \text{y} \quad
	P = \mat{\vec{v}_1 & \cdots & \vec{v}_r 
		& \Re \vec{w}_1 & \Im \vec{w}_1 & 
		\cdots & \Re \vec{w}_s & \Im \vec{w}_s},\]
siendo \(\vec{v}_j\) y \(\vec{w}_j\) autovectores asociados a \(\lambda_j\) 
y \(a_j + \iu b_j\), respectivamente.

\begin{theorem}
	El sistema \(\vec{x}' = A\vec{x}\) tiene solución general real:
	\[\vec{x}(t) = P
		\mat{e^{\lambda_1 t} & & & & & \\
			& \ddots & & & & \\
			& & e^{\lambda_r t} & & & \\
			& & & e^{B_1 t} & & \\
			& & & & \ddots & \\
			& & & & & e^{B_s t}}
		\mat{c_1 \\ \vdots \\ c_r \\ k_1 \\ k_2 \\ \vdots \\ k_{2s-1}\\ k_{2s}},
		\quad 
		\mat{c_1 \\ \vdots \\ c_r \\ k_1 \\ k_2 \\ \vdots \\ k_{2s-1}\\ k_{2s}}
		\in \R^n,\]
	donde
	\[e^{B_j t} := \mat{e^{a_j t} \cos b_j t & e^{a_j t} \sin b_j t \\
		-e^{a_j t} \sin b_j t & e^{a_j t} \cos b_j t}.\]
\end{theorem}

\begin{proof}
	Puesto que los autovectores complejos siempre se presentan en
	pares obtendremos una matriz semejante a la original diagonal por bloques
	\(D\); la matriz de cambio de paso que nos permite hacer esto es \(P\), 
	sin más que multiplicar:
	\[D := \mat{\lambda_1 & & & & & & & \\
		& \ddots & & & & & & \\
		& & \lambda_r & & & & & \\
		& & & a_1 & b_1 & & & & \\
		& & & -b_1 & a_1 & & & & \\
		& & & & & \ddots & & \\
		& & & & & & a_s & b_s \\
		& & & & & & -b_s & a_s} = P^{-1}AP.\]
	Para resolver el sistema basta ahora obtener la exponencial de esta matriz,
	pero esto equivale a computar la exponencial de cada elemento de la diagonal,
	por bloques, igual que en el caso diagonal (¿por qué?):
	\[e^{A t} = P e^{D t} P^{-1} =
		\mat{e^{\lambda_1 t} & & & & & \\
			& \ddots & & & & \\
			& & e^{\lambda_r t} & & & \\
			& & & e^{B_1 t} & & \\
			& & & & \ddots & \\
			& & & & & e^{B_s t}}.\]
\end{proof}

Vemos ahora un ejemplo concreto de cómo aplicar este teorema.

\begin{example}
	Resolver el sistema \(\vec{x}' = A\vec{x}\), siendo
	\[A = \mat{1 & 0 & 0 \\ 6 & 2 & -3 \\ 1 & 3 & 2}.\]
\end{example}

\begin{solution}
	En primer lugar hay que diagonalizar la matriz: sus autovalores son
	\(\lambda_1 = 1\), \(\lambda_2 = 2 + 3\iu\) y \(\lambda_3 = 2 - 3\iu\), 
	con espacios propios asociados
	\[\ker(A - \lambda_1 I) = L \left[ \mat{10 \\ 3 \\ -19} \right], \quad
		\ker(A - \lambda_2 I) = L \left[ \mat{0 \\ 1 \\ -\iu} \right], \quad
		\ker(A - \lambda_3 I) = L \left[\mat{0 \\ 1 \\ \iu} \right],\]
	con lo que podemos tomar
	\[v_1 = \mat{10 \\ 3 \\ -19}, 
		\quad 
		v_2 = \Re \mat{0 \\ 1 \\ -\iu} 
		= \mat{0 \\ 1 \\ 0}, 
		\quad 
		v_3 = \Im \mat{0 \\ 1 \\ -\iu} = \mat{0 \\ 0 \\ -1}.\]
	Así, obtenemos la matriz de paso \(P\) tal que
	\[P = \mat{10 & 0 & 0 \\ 3 & 1 & 0 \\ -19 & 0 & -1}
		\quad \text{y} \quad
		P^{-1}AP = \mat{1 & & \\ & 2 & 3 \\ & -3 & 2},\]
	que no es diagonal pero sí diagonal por bloques, lo que nos permite tomar 
	la exponencial, por bloques:
	\[A = P \mat{1 & & \\ & 2 & 3 \\ & -3 & 2} P^{-1}
		\quad \text{y} \quad
		e^{At} = P \mat{e^t & & \\ & e^{2t} \cos 3t & e^{2t} \sin 3t \\ &
		-e^{2t} \sin 3t & e^{2t} \cos 3t} P^{-1}.\]
	La solución general del sistema es entonces
	\[\vec{x}(t) = P \mat{e^t & & \\ & e^{2t} \cos 3t & e^{2t} \sin 3t \\ &
		-e^{2t} \sin 3t & e^{2t} \cos 3t} \mat{c_1 \\ c_2 \\ c_3}, \quad
		\mat{c_1 \\ c_2 \\ c_3} \in \R^3.\]
\end{solution}

\subsection{Matrices no diagonalizables}

Para terminar con el análisis de los sistemas de ecuaciones diferenciales
lineales con coeficientes constantes, únicamente nos falta estudiar qué ocurre
cuando la matriz que define el sistema no es diagonalizable.

\begin{theorem}[Forma canónica de Jordan compleja]
	Sea \(A \in \mathcal{M}_{n \times n}(\Complex)\) con \(s\) autovectores
	linealmente independientes. Entonces, existe una matriz no singular \(P\)
	tal que \(P^{-1} A P = B\), donde \(B\) es una matriz diagonal por bloques,
	es decir,
	\[B = \mat{B_1 & & & \\ & B_2 & & \\ & & \ddots & \\ & & & B_s},\]
	donde cada \emph{bloque de Jordan} \(B_j\ (j = 1, \dots, s)\) es una matriz de
	la forma
	\[B_j = \mat{\lambda & 1 & & \\ & \lambda & \ddots & \\ & & \ddots & 1 \\ & & &
			\lambda},\]
	siendo \(\lambda\) autovalor de \(A\).
\end{theorem}

\begin{remark}
	Cada autovalor de \(A\) apararece en \(B\) tantas veces como su
	multiplicidad algebraica. El número de bloques asociados a un mismo
	autovalor se corresponde con su multiplicidad geométrica.
\end{remark}

\begin{remark}
	El caso diagonalizable corresponde al caso en que todos los bloques son \(1
	\times 1\).
\end{remark}

Veamos cómo calcular la exponencial de un bloque de Jordan arbitrario:

\[B_j = \mat{\lambda & 1 & & \\ & \lambda & \ddots & \\ & & \ddots & 1 \\ & &
		& \lambda} =
	\underbrace{\mat{\lambda & & & \\ & \lambda & & \\ & & \ddots & \\ & & & \lambda}}_D
	+ \underbrace{\mat{0 & 1 & & \\ & 0 & \ddots & \\ & & \ddots & 1 \\ & & & 0}}_N\]

Ya sabemos calcular \(e^{Dt}\), y hallar \(e^{Nt}\) tampoco es complicado. Sus
potencias sucesivas son

\[
	N^0 = \mat{1 & 0 & 0 & \cdots & 0      \\
		& 1 & 0 & \ddots & \vdots \\
		&   & 1 & \ddots & 0      \\
		&   &   & \ddots & 0      \\
		&   &   &        & 1 }, \quad
	N^1 = \mat{0 & 1 & 0 & \cdots & 0      \\
		& 0 & 1 & \ddots & \vdots \\
		&   & 0 & \ddots & 0      \\
		&   &   & \ddots & 1      \\
		&   &   &        & 0 }, \ \ \dots \ \ ,
	N^{n_j-1} = \mat{0 & 0 & 0 & \cdots & 1      \\
		& 0 & 0 & \ddots & \vdots \\
		&   & 0 & \ddots & 0      \\
		&   &   & \ddots & 0      \\
		&   &   &        & 0 }
\]

y, en particular, \(N\) es nilpotente de orden \(n_j\), por lo que basta
aplicar la definición para obtener

\[e^{Nt} = \sum_{k=0}^\infty \frac{t^k}{k!}N^k = \sum_{k=0}^{n_j-1}
	\frac{t^k}{k!}N^k =
	\mat{1 & t & \frac{t^2}{2!} & \cdots & \frac{t^{n_j-1}}{(n_j-1)!} \\
		& 1 & t              & \ddots & \vdots                    \\
		&   & 1              & \ddots & \frac{t^2}{2!}            \\
		&   &                & \ddots & t                         \\
		&   &                &        & 1 }
\]

Finalmente, como \(D = \lambda I\) y \(N\) conmutan, se tiene

\[e^{B_j t} = e^{Dt + Nt} = e^{Dt} e^{Nt} = (e^{\lambda t} I) e^{Nt} =
	e^{\lambda t}e^{Nt} =
	\mat{e^{\lambda t} & te^{\lambda t} & \frac{t^2}{2!}e^{\lambda t} & \cdots & \frac{t^{n_j-1}}{(n_j-1)!}e^{\lambda t} \\
	& e^{\lambda t} & te^{\lambda t}              & \ddots & \vdots                    \\
	&   & e^{\lambda t}              & \ddots & \frac{t^2}{2!}e^{\lambda t}            \\
	&   &                & \ddots & te^{\lambda t}                         \\
	&   &                &        & e^{\lambda t} }.\]

Esto es suficiente en el caso de que \(A\) sólo tenga autovalores reales. Si
hay alguno complejo, hay que trabajar un poco más:

\begin{theorem}[forma canónica de Jordan real]
	Sea \(A \in \mathcal{M}_{n \times n}(\R)\). Existe una matriz no singular
	\(P\) tal que \(P^{-1}AP = B\), donde \(B\) es una matriz diagonal por
	bloques. Dependiendo de sus autovalores estos bloques pueden tomar dos
	formas:
	\begin{enumerate}[align=left]
		\item[si \(\lambda\) es un autovalor real de \(A\):] 
			\[B_\lambda = \mat{\lambda & 1 & & \\ & \lambda & \ddots & \\ & & \ddots & 1 \\ & & & \lambda}.\]

		\item[si \(\lambda = a+ib \ (b>0)\) es un autovalor complejo de \(A\):]
			\[B_\lambda = 
				\mat{D & I_2 & & \\ & D & \ddots & \\ & & \ddots & I_2 \\ & & & D}
				= \mat{a & b & 1 & 0 \\ -b & a & 0 & 1 
					\\ && a & b & 1 & 0 \\ && -b & a & 0 & 1
					\\ &&&&& \ddots & \ddots 
					 \\ &&&&&& a & b & 1 & 0 \\ &&&&&& -b & a & 0 & 1}.\]
	\end{enumerate}
\end{theorem}

A la vista de este teorema, sólo nos falta saber calcular la exponencial de
matrices de la forma

\[\tilde{D} = \mat{D & I_2 & & \\ & D & \ddots & \\ & & \ddots & I_2 \\ & & & D} =
	\mat{D & & & \\ & D & & \\ & & \ddots & \\ & & & D} +
	\underbrace{\mat{0 & I_2 & & \\ & 0 & \ddots & \\ & & \ddots & I_2 \\ & & &
			0}}_{N}.\]

Observamos que estas dos matrices conmutan:
\[\mat{D & & & \\ & D & & \\ & & \ddots & \\ & & & D} N =
	\mat{0 & D & & \\ & 0 & \ddots & \\ & & \ddots & D \\ & & & 0}
	= N \mat{D & & & \\ & D & & \\ & & \ddots & \\ & & & D},\]

luego la exponencial de su suma es
\[\exp(\tilde{D}) t =
	\exp \mat{Dt & & & \\ & Dt & & \\ & & \ddots & \\ & & & Dt} \cdot \exp(Nt)
	= \mat{e^{at}R & & & \\ & e^{at}R & & \\ & & \ddots & \\ & & & e^{at}R}
	\cdot \exp(Nt),\]

puesto que 
\[\exp\mat{Dt & & & \\ & Dt & & \\ & & \ddots & \\ & & & Dt} =
	\mat{e^{Dt} & & & \\ & e^{Dt} & & \\ & & \ddots & \\ & & & e^{Dt}}
	\quad \text{y} \quad
	e^{Dt} = e^{at} \underbrace{\mat{\cos bt & \sin bt \\ -\sin bt & \cos
		bt}}_{R}.\]

Por otro lado, igual que antes es fácil ver que
\[e^{Nt} = \sum_{k=0}^{n_j-1} \frac{t^k}{k!}N^k =
	\mat{I_2 & tI_2 & \frac{t^2}{2!}I_2 & \cdots & \frac{t^{n_j-1}}{(n_j-1)!}I_2 \\
		& i_2  & tI_2              & \ddots & \vdots                        \\
		&      & I_2               & \ddots & \frac{t^2}{2!}I_2             \\
		&      &                   & \ddots & tI_2                          \\
		&      &                   &        & I_2 }.\]

Juntándolo todo, llegamos a
\[\exp(\tilde{D}) t =
	\mat{e^{\lambda t}R & te^{\lambda t}R & \frac{t^2}{2!}e^{\lambda t}R & \cdots & \frac{t^{n_j-1}}{(n_j-1)!}e^{\lambda t}R \\
	& e^{\lambda t}R & te^{\lambda t}R              & \ddots & \vdots                    \\
	&   & e^{\lambda t}R              & \ddots & \frac{t^2}{2!}e^{\lambda t}R            \\
	&   &                & \ddots & te^{\lambda t}R                         \\
	&   &                &        & e^{\lambda t}R }
\]

Ahora que ya sabemos resolver cualquier sistema con coeficientes constantes,
centrémonos en cómo calcular la forma canónica de Jordan. La clave será
encontrar propiedades de \(B\) que caractericen su estructura de bloques y que
sean invariantes por semejanza, para poder hallarlas a partir de \(A\).

Por ser \(P\) no singular, se cumple \(\dim \ker (B - \lambda I) = \dim \ker
(P^{-1}(A - \lambda I)P) = \dim \ker (A - \lambda I)\) o, equivalentemente,
\(\ran (A - \lambda I) = \ran (B - \lambda I)\).

Recordando la estructura de \(B\), es fácil ver que
\(\dim \ker (B - \lambda I) = \sum_{i=1}^r \nu_i(\lambda)\), siendo
\(\nu_i(\lambda)\) el número de bloques de tamaño \(i \times i\) del autovalor
\(\lambda\).

Consideramos ahora potencias sucesivas de la matriz \(B - \lambda I\). Al
elevar al cuadrado, la dimensión del núcleo aumenta en una unidad por cada
bloque de tamaño mayor que 1:

\[
	B_j - \lambda I =
	\mat{0 & 1 &   &        &        &   \\
		& 0 & 1 &        &        &   \\
		&   & 0 & 1      &        &   \\
		&   &   & \ddots & \ddots &   \\
		&   &   &        & \ddots & 1 \\
		&   &   &        &        & 0 \\} \qquad
	(B_j - \lambda I)^2 =
	\mat{0 & 0 & 1      &        &        &   \\
		& 0 & 0      & 1      &        &   \\
		&   & \ddots & \ddots & \ddots &   \\
		&   &        & \ddots & \ddots & 1 \\
		&   &        &        & \ddots & 0 \\
		&   &        &        &        & 0 \\},
\]
por lo que
\(\dim \ker (B - \lambda I)^2 = \nu_1(\lambda) + 2 \sum_{i=2}^r
\nu_i(\lambda)\). Análogamente, para las terceras potencias se tiene
\(\dim \ker (B - \lambda I)^3 = \nu_1(\lambda) + 2\nu_2(\lambda) + 3
\sum_{i=3}^r \nu_i(\lambda)\) y, en general, para \(k = 1, \dots, r\):
\begin{align*}
	\dim \ker (B - \lambda I)^k & = \sum_{i=1}^{k-1} i \nu_i(\lambda) + k
	\sum_{i=k}^r \nu_i(\lambda)                                                         \\
	                            & = \nu_1(\lambda) + \cdots + (k-1)\nu_{k-1}(\lambda) +
	k[\nu_k(\lambda) + \cdots + \nu_r(\lambda)]
\end{align*}

\begin{remark}
	\(P(B - \lambda I)^kP^{-1} = (A - \lambda I)^k\) y, como el rango es
	invariante por semejanza, \(\dim \ker (A - \lambda I)^k = \dim \ker (B -
	\lambda I)^k\).
\end{remark}

\begin{remark}
	Los números \(\nu_i(\lambda)\) caracterizan \(B\), salvo permutación de bloques.
\end{remark}

Denotamos \(\delta_i(\lambda) = \dim \ker (A - \lambda I)^i\). De esta forma,
\(r\) es el menor número natural tal que \(\delta_r(\lambda) =
\delta_{r+1}(\lambda)\) y se verifica

\[\left\{
	\begin{array}[ht]{r c l}
		\delta_1(\lambda) & =      & \nu_1(\lambda) + \nu_2(\lambda) + \cdots +
		\nu_r(\lambda)                                                                           \\
		\delta_2(\lambda) & =      & \nu_1(\lambda) + 2[\nu_2(\lambda) + \cdots +
		\nu_r(\lambda)]                                                                          \\
		                  & \vdots &                                                             \\
		\delta_r(\lambda) & =      & \nu_1(\lambda) + 2\nu_2(\lambda) + \cdots + r\nu_r(\lambda)
	\end{array}
	\right.
\]

Restando a cada ecuación la anterior, queda un sistema triangular

\[\left\{
	\begin{array}[ht]{r c c c c c c c c}
		\delta_1(\lambda)                          & =      & \nu_1(\lambda) & + & \nu_2(\lambda) & + & \cdots & + & \nu_r(\lambda) \\
		-\delta_1(\lambda) + \delta_2(\lambda)     & =      &                &   & \nu_2(\lambda) & + & \cdots & + & \nu_r(\lambda) \\
		                                           & \vdots &                                                                       \\
		-\delta_{r-1}(\lambda) + \delta_r(\lambda) & =      &                &   &                &   &        &   & \nu_r(\lambda)
	\end{array}
	\right.
\]

que tiene por solución

\[\left\{
	\begin{array}[ht]{r c l}
		\nu_1(\lambda) & = & 2\delta_1(\lambda) - \delta_2(\lambda)                              \\
		\nu_k(\lambda) & = & -\delta_{k-1}(\lambda) + 2\delta_k(\lambda) - \delta_{k+1}(\lambda)
	\end{array}
	\right.
\]

Pasar de la forma de Jordan compleja a la real se hace como uno podría
esperar:

\[B_\Complex =
	\mat{\ddots &      &      &      &      &         \\
		& a+bi & 1    &      &      &         \\
		& 0    & a+bi &      &      &         \\
		&      &      & a-bi & 1    &         \\
		&      &      & 0    & a-bi &         \\
		&      &      &      &      & \ddots} \leadsto
	B_\R =
	\mat{\ddots &    &   &    &      &         \\
		& a  & b & 1  & 0 &         \\
		& -b & a & 0  & 1 &         \\
		&    &   & a  & b &         \\
		&    &   & -b & a &         \\
		&    &   &    &   & \ddots},
\]

pero esto requiere modificar también la matriz de paso compleja:

\[P_\Complex =
	\mat{\cdots & v_i & v_{i+1} & \cdots} \leadsto
	P_\R =
	\mat{\cdots & \Re v_i & \Im v_i & \cdots}\]

Falta únicamente ver cómo calcular
\(P_\Complex = \mat{v_1 & v_2 & \cdots & v_n}\). Para ello, es suficiente
estudiar cómo afecta \(B\) a los vectores de la base canónica, puesto que
\(A\) y la base \(\{v_i : i = 1, \dots, n\}\) se obtienen con el cambio de
base dado por \(P_\Complex\). Si el bloque \(j\)-ésimo de la matriz \(B\) (de
tamaño \(n_j \times n_j\)) empieza en el índice \(l+1\), se tiene

\[\left\{
	\begin{array}[ht]{r c l}
		Av_{l+1}    & =      & \lambda v_{l+1}                 \\
		Av_{l+2}    & =      & \lambda v_{l+2} + v_{l+1}       \\
		            & \vdots &                                 \\
		A v_{l+n_j} & =      & \lambda v_{l+n_j} + v_{l+n_j-1}
	\end{array} \right. \iff
	\left\{
	\begin{array}[ht]{r c l}
		Av_{l+1} & = & \lambda v_{l+1}                               \\
		Av_{l+k} & = & \lambda v_{l+k} + v_{l+k-1}, \ 1 < k \leq n_j \\
	\end{array}
	\right.\]

Estas ecuaciones (junto con el hecho de que \(P\) es invertible)
caracterizan \(P\).
\end{document}
